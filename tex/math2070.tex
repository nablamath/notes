% ~~~~~~~~~~~~~~~~~~~~~~~~~~~~~~~~~~~~~~~~~~~~~~~~~~~~~~~~~~~~~~
%
% CUHK Mathematics
% MATH2070: Algebraic Structures
%
% ~~~~~~~~~~~~~~~~~~~~~~~~~~~~~~~~~~~~~~~~~~~~~~~~~~~~~~~~~~~~~~

\documentclass[a4paper,12pt]{article}
\usepackage{standalone}
% Basic
\usepackage[T1]{fontenc}
\usepackage[utf8]{inputenc}

% ------------------------------------------------------------ %

% Packages
\usepackage{amsmath, amsthm, amssymb}
\usepackage{blindtext}
\usepackage{enumitem}
\usepackage{extramarks}
\usepackage{fancyhdr}
\usepackage[margin=1in]{geometry}
\usepackage{graphicx}
\usepackage{hyperref}
\usepackage{indentfirst}
\usepackage{listings}
\usepackage{mathrsfs}
\usepackage{mdframed}
\usepackage{multicol, multirow}
\usepackage{needspace, setspace}
\usepackage{paracol}
\usepackage{pgf, tikz, pgfplots}
\usepackage{silence}
\usepackage{xcolor}

% ------------------------------------------------------------ %

% Spacings
\newcommand{\n}{\vspace{3mm}} % Context spacing
\newcommand{\s}{\vspace{1mm}} % Equation spacing
\newcommand{\m}{\vspace{-3mm}} % Reverse context spacing
\newcommand{\propdisp}{\pagebreak} % Proper display page break

% Wordings
\newcommand{\ans}[1][zb]{{\color{#1}\textit{Answer. }\hspace{3mm}}} % Answer
\newcommand{\arl}[1][zr]{{\color{#1}$\brr{\Leftarrow}$\hspace{3mm}}} % Left arrow
\newcommand{\arr}[1][zr]{{\color{#1}$\brr{\Rightarrow}$\hspace{3mm}}} % Right arrow
\newcommand{\cse}[2][zr]{{\color{#1}\textit{Case #2: }\hspace{1mm}}} % Case
\newcommand{\clm}[2][zr]{{\color{#1}$\vdash_{#2}$\hspace{1mm}}} % Claim
\newcommand{\prf}[1][zr]{{\color{#1}\textit{Proof. }\hspace{3mm}}} % Proof
\newcommand{\prt}[2][a]{\hspace{-2mm}{\color{#2}\textit{Part (#1) }}\hspace{1mm}} % Part
\newcommand{\prtc}[2][a]{\hspace{2mm}\prt[#1]{#2}} % Continued part

\newcommand{\rdft}[1][\sct]{{\color{zg}\textit{Definition #1}}} % Refer definition
\newcommand{\rexm}[1][\sct]{{\color{zb}\textit{Example #1}}} % Refer example
\newcommand{\rfig}[1][\sct]{{\color{zy}\textit{Figure #1}}} % Refer figure
\newcommand{\rpst}[1][\sct]{{\color{zr}\textit{Proposition #1}}} % Refer proposition
\newcommand{\rthm}[1][\sct]{{\color{zr}\textit{Theorem #1}}} % Refer theorem

\newcommand{\sct}{\thesection.\thescount} % Counter
\newcommand{\sctd}[2][0]{\the\numexpr\value{section}-#1\relax.\the\numexpr\value{scount}-#2\relax} % Delta counter

% Equations
\newcommand{\C}{\mathbb{C}} % Complex
\newcommand{\F}{\mathbb{F}} % Field
\newcommand{\I}{\mathbb{I}} % Irrational
\newcommand{\N}{\mathbb{N}} % Natural
\newcommand{\Q}{\mathbb{Q}} % Rational
\newcommand{\R}{\mathbb{R}} % Real
\newcommand{\Z}{\mathbb{Z}} % Integer

\newcommand{\GL}{\mathrm{GL}} % General linear group
\newcommand{\SL}{\mathrm{SL}} % Speical linear group

\newcommand{\abs}[1]{\left| #1\right|} % Absolute
\newcommand{\bra}[1]{\left\langle #1\right\rangle} % Angled brackets
\newcommand{\brc}[1]{\left\{ #1\right\}} % Curly brackets
\newcommand{\brr}[1]{\left( #1\right)} % Round brackets
\newcommand{\brs}[1]{\left[ #1\right]} % Square brackets
\newcommand{\cond}[1]{\left. #1\right|} % Condition with right bar
\newcommand{\diff}{\,\mathrm{d}} % Differential
\newcommand{\erm}[1]{\;\;\;\;\text{#1}} % Equation remarks
\newcommand{\nrm}[1]{\left\| #1\right\|} % Norm
\newcommand{\srm}{\,\mid\,} % Set remarks

% Math operators
\let\Im\relax
\let\Re\relax

\DeclareMathOperator{\Im}{Im} % Imaginary function
\DeclareMathOperator{\Re}{Re} % Real function

% ------------------------------------------------------------ %

% Colours
\definecolor{gr}{RGB}{120, 120, 120} % Grey
\definecolor{zb}{RGB}{0, 38, 77} % Blue
\definecolor{zg}{RGB}{0, 77, 51} % Green
\definecolor{zp}{RGB}{51, 0, 77} % Purple
\definecolor{zr}{RGB}{77, 0, 38} % Red
\definecolor{zy}{RGB}{77, 64, 0} % Yellow

% Graphing
\usetikzlibrary{arrows}
\usetikzlibrary{calc}
\usetikzlibrary{patterns}
\pgfplotsset{compat=1.15}

% ------------------------------------------------------------ %

% Remark
\newcommand{\remark}[1]{
  \noindent\textbf{Remarks}
  
  \begin{nlist}
    \item Context of this document is based on university course \textit{\gettitle} from \textit{Department of Mathematics, The Chinese University of Hong Kong (CUHK)}. The original source can be found at \url{https://www.math.cuhk.edu.hk/course}. The author does not own the source.
    \item This document is assumed unavailable for unauthorized parties that have not attended the university course. It is prohibited to share, including distributing or copying this document to unauthorized parties in any means for any non-academic purpose.
    \item Context of this doucment may not be completely accurate. The author assumes no responsibility or liability for any errors or omissions in the context of this document.
    \item This document is under license CC-BY-SA 4.0. It is allowed to make any editions on this document, as long as terms of the license is not violated.
    #1
  \end{nlist}
}

% Prerequisites
\newenvironment{prereq}{
\noindent \textbf{Prerequisites}\n

This course requires prerequisites of
}{
\n
}

% Reference list
\newenvironment{reflist}{
\begin{alist}
  \item Professor(s) associated to \textit{\gettitle}
}{
\end{alist}
}

% ------------------------------------------------------------ %

% Environments
\newcounter{scount}[section] % Counter

\newenvironment{crl}{ % Corollary
\parindent0pt
\begin{siderule}[linecolor=zr]{\color{zr}\textit{Corollary. }}
}{
\end{siderule}
}

\newenvironment{cmt}{ % Comment
\parindent0pt
\begin{siderule}[linecolor=zp]{\color{zp}\textit{Comment. }}
}{
\end{siderule}
}

\newenvironment{dft}{ % Definition
\parindent0pt
\refstepcounter{scount}
\begin{siderule}[linecolor=zg]{\color{zg}\textit{Definition \sct. }}
}{
\end{siderule}
}

\newenvironment{exm}{ % Example
\parindent0pt
\refstepcounter{scount}
\begin{siderule}[linecolor=zb]{\color{zb}\textit{Example \sct. }}
}{
\end{siderule}
}

\newenvironment{fig}{ % Figure
\parindent0pt
\refstepcounter{scount}
\begin{siderule}[linecolor=zy]{\color{zy}\textit{Figure \sct. }}\n

}{
\end{siderule}
}

\newenvironment{prv}{ % Proof
\parindent0pt
\begin{siderule}[linecolor=zr]\prf
}{
\end{siderule}
}

\newenvironment{pst}{ % Proposition
\parindent0pt
\refstepcounter{scount}
\begin{siderule}[linecolor=zr]{\color{zr}\textit{Proposition \sct. }}
}{
\end{siderule}
}

\newenvironment{tcn}{ % Technique
\parindent0pt
\begin{siderule}[linecolor=zp]{\color{zp}\textit{Technique. }}
}{
\end{siderule}
}

\newenvironment{thm}{ % Theorem
\parindent0pt
\refstepcounter{scount}
\begin{siderule}[linecolor=zr]{\color{zr}\textit{Theorem \sct. }}
}{
\end{siderule}
}

\newenvironment{rmatrix}{ % Matrix in round brackets
  \left( \begin{matrix}
}{
  \end{matrix}\right)
}

\newenvironment{alist}{ % Alphabetical list
\begin{enumerate}[label=(\alph*)]
}{
\end{enumerate}
}

\newenvironment{Alist}{ % Capitalized alphabetical list
\begin{enumerate}[label=(\Alph*)]
}{
\end{enumerate}
}

\newenvironment{nlist}{ % Number list
\begin{enumerate}[label=(\arabic*)]
}{
\end{enumerate}
}

\newenvironment{plist}{ % Point list
\begin{itemize}
}{
\end{itemize}
}

\newenvironment{rlist}{ % Roman list
\begin{enumerate}[label=(\roman*)]
}{
\end{enumerate}
}

\newmdenv[ % Siderule line
  topline=false,
  bottomline=false,
  rightline=false,
  rightmargin=0
]{siderule}

% ------------------------------------------------------------ %

% Warning filters
\WarningFilter{mdframed}{You got a bad break}
\hfuzz=8pt


\begin{document}
\title{MATH2070: Algebraic Structures}
\author{onenylxus/math-notes}
\makeatletter
\let\getauthor\@author
\let\gettitle\@title
\makeatother
\maketitle
\thispagestyle{empty}
\n\n

\noindent \textbf{Remarks}


\remark{}

\tableofcontents
\thispagestyle{empty}
\pagebreak
\pagestyle{fancy}
\fancyhf{}
\setlength{\headheight}{15.2pt}
\fancyhead[R]{\nouppercase \lastrightmark}
\fancyfoot[L]{\gettitle}
\fancyfoot[R]{\thepage}


\section{Group Theory}
\subsection{Groups}
\subsubsection{Definition of Groups}
To start off with group theory, below is the definition of groups:\n

\begin{dft}
  A \textbf{group} $(G,*)$ is a set $G$ equipped with a binary operation

  $$*:G\times G\rightarrow G$$\s

  which can be called as the \textbf{group operation/product/multiplication} such that the following conditions are satisfied:

  \begin{alist}
    \item The group operation is associative, which is

    $$(a*b)*c=a*(b*c)\erm{for all }a,b,c\in G$$

    \item There exists an \textbf{identity element} $e\in G$ such that

    $$a*e=e*a=a\erm{for all }a\in G$$

    \item For every $a\in G$, there exists an element $a^{-1}\in G$, which is the \textbf{inverse} of $a$, such that

    $$a*a^{-1}=a^{-1}*a=e$$
  \end{alist}
\end{dft}\n

Note that it is also allowed to express the group operation in $ab$ or $a\cdot b$.\n

\begin{dft}
  The group $(G,*)$ is \textbf{abelian} if the group operation $*$ is commutative, which is

  $$a*b=b*a\erm{for all }a,b\in G$$\s

  Otherwise, the group is \textbf{nonabelian}.
\end{dft}\n

Note that if the group $(G,*)$ is abelian, the sign of the group operation $*$ is often replaced by $+$.\n

\begin{dft}
  The \textbf{order} of the group $(G,*)$, denoted by $\abs{G}$, is the cardinality of the set $G$. The set $G$ is said to be \textbf{finite} if the order of the group $\abs{G}$ is finite, or otherwise it is said to be \textbf{infinite}.
\end{dft}\n

Below are some examples on identifying groups:\n

\begin{exm}
  Let $n\in\Z^{+}$, then

  $$\begin{aligned}[t]
    U_{n}&:=\brc{z\in\C\srm z^{n}=1}\\
    &=\brc{e^{\frac{2\pi ik}{n}}\srm 0\leq k\leq n-1}
  \end{aligned}$$\s

  is an abelian group under multiplication of complex numbers.\n

  \prf[zb] The well-definedness of the group operation has to be checked, thus first let $a,b\in U_{n}$, then

  $$a^{n}=b^{n}=1\\
  \Rightarrow(ab)^{n}=a^{n}\cdot b^{n}=1\cdot 1=1\\
  \Rightarrow a\cdot b\in U_{n}$$\s

  After that, the three conditions of a group are checked one by one:

  \begin{alist}
    \item Associtativity follows from that of multiplication of complex numbers.
    \item An identity element is given by $1=e^{\frac{2\pi i\cdot 0}{n}}\in U_{n}$.
    \item Given $a\in U_{n}$, $a=e^{\frac{2\pi ik}{n}}$ for some $0\leq k\leq n-1$. Then $\frac{1}{a}=e^{\frac{2\pi i(n-k)}{n}}$ is an inverse to $a$.
  \end{alist}

  Hence $(U_{n},\cdot)$ is a group. It follows from the fact that multiplication in $\C$ is commutative, which means $(U_{n},\cdot)$ is abelian.
\end{exm}\n

\begin{exm}
  Let $X$ be a nonempty set. The set of all symmetries (or permutations) of $X$ is defined as

  $$S_{X}:=\brc{\rho:X\rightarrow X\srm\rho\text{ is bijective}}$$\s

  Then $S_{X}$ forms a group under composition of maps, for example

  $$(\rho_{1},\rho_{2})\in S_{X}\mapsto \rho_{1}\circ\rho_{2}\in S_{X}$$\s

  \prf[zb] Again, the three conditions of a group are checked:

  \begin{alist}
    \item Composition of maps is associative.
    \item The identity map $Id:x\mapsto x$ is an identity element.
    \item $\rho\in S_{X}$ is bijective, which means $\rho^{-1}$ exists and is also bijective, which is an element of $S_{X}$.
  \end{alist}

  Note that the group above is finite nonabelian group.
\end{exm}

\propdisp

\subsubsection{Linear Groups}
\begin{dft}
  The \textbf{General Linear Group} of $n\times n$ real matrices $M$, such that $\det(M)\neq 0$, and denoted by $\mathrm{GL}(n,\R)$, is a group under matrix multiplication.
\end{dft}\n

\begin{dft}
  The \textbf{Special Linear Group} of $n\times n$ real matrices $M$, such that $\det(M)=1$, and denoted by $\mathrm{SL}(n,\R)$, is a group under matrix multiplication.
\end{dft}

\subsubsection{Basic Properties of Groups}
\begin{pst}
  The identity element in a group is unique.\n

  \prf Suppose $e,e'\in G$ are two identity elements in $G$, then

  $$e'=e'\cdot e=e$$

  Therefore $e'=e$.
\end{pst}\n

\begin{pst}
  Let $G$ be a group, then for any $a\in G$, inverse of $a$ is unique.\n

  \prf Suppose $a',a''\in G$ are two inverses of $a$, then

  $$\begin{aligned}[t]
    a''&=a''\cdot e\\
    &=a''\cdot(a\cdot a')\\
    &=(a''\cdot a)\cdot a'\\
    &=e\cdot a'\\
    &=a'
  \end{aligned}$$

  Therefore $a'=a''$.
\end{pst}\n

\begin{pst}
  Let $G$ be a group, then $(ab)^{-1}=b^{-1}a^{-1}\erm{for any }a,b\in G$.\n

  \prf Note that $(ab)(b^{-1}a^{-1})=(b^{-1}a^{-1})(ab)=e$, along with the uniqueness of inverse (by \rpst[\sctd{2}]), $(ab)^{-1}=b^{-1}a^{-1}$.
\end{pst}\n

\begin{dft}
  Let $G$ be a group with identity element $e$, for any $g\in G$, $n\in\N$, let

  $$g^{n}=\begin{cases}
    \underset{k\text{ times}}{\underbrace{g\cdots g}} &\erm{, if }k\in\Z^{+}\\
    $e$ &\erm{, if }k=0\\
    \underset{k\text{ times}}{\underbrace{g^{-1}\cdots g^{-1}}} &\erm{, if }k\in\Z_{<0}\\
  \end{cases}$$
\end{dft}\n

\begin{dft}
  The \textbf{Klein-4 group} is defined by $\Z_{2}\times\Z_{2}$ with elements $(0,0),(0,1),(1,0)$ and $(1,1)$.
\end{dft}

\subsection{Cyclic Groups}
\subsubsection{Element Order}
\begin{dft}
  Let $G$ be a group with identity element $e\in G$. The \textbf{order} of an element $g\in G$, denoted by $\abs{g}$, is the smallest positive integer $n$ such that $g^{n}=e$ if such integer exist. Otherwise, $g$ has infinite order, which is $\abs{g}=\infty$.
\end{dft}\n

In other words, the order of an element can be found as follows:

$$\abs{g}=\begin{cases}
  \min\brc{k\in\Z^{+}\srm g^{k}=e}&\erm{, if such set is nonempty}\\
  \infty &\erm{, otherwise}
\end{cases}$$\s

\begin{pst}
  Let $G$ be a group with identity element $e\in G$, then for any $n\in\Z^{+}$ such that $g^{n}=e$ where $g\in G$, $\abs{g}\mid n$.\n

  \prf Let $m:=\abs{g}$ (which is finite by assumption). Suppose $g^{n}=e$ for some $n\in\Z^{+}$. By the Division Algorithm in $Z$, there exists some $q,r\in\Z$ with $0\leq r\leq m$ such that

  $$\begin{aligned}[t]
    &n=qm+r\\
    \Rightarrow\; &e=g^{n}=g^{qm+r}=\brr{g^{m}}^{q}g^{r}=g^{r}
  \end{aligned}$$\s

  which means that $r=0$ (otherwise there will be a contradiction based on the definition of $m=\abs{g}$). Therefore $n=qm$, or $m\mid n$.
\end{pst}\n

\begin{thm}
  Let $G$ be a group. Fix an element $g\in G$. If $\abs{g}=\infty$, then $\bra{g}$ is an infinite countable set. If $\abs{g}=m<\infty$, then

  $$\bra{g}=\brc{e,g,g^{2},\cdots,g^{m-1}}$$\s

  \prf Suppose $\abs{g}=\infty$, then by defintion, the map $\varphi:\Z\rightarrow\bra{g}$ is surjective. Now if $g^{k_{1}}=g^{k_{2}}$ for $k_{1},k_{2}\in\Z$, and without the loss of generality, assume that $k_{1}\geq k_{2}$. Then $g^{k_{1}-k_{2}}=e$ and $\abs{g}=\infty\Leftrightarrow\brc{k\in\Z^{+}\;\mid\; g^{k}=e}=\phi$. Hence $\varphi$ is injective.\n

  Suppose $\abs{g}=m<\infty$, then it has to show that $\bra{g}\subset\brc{e,g,g^{2},\cdots,g^{m-1}}$. Let $g^{k}\in\bra{g}$ where $k\in\Z$. By \rpst[\sctd{1}], $g^{r}\in\brc{e,g,g^{2},\cdots,g^{m-1}}$.
\end{thm}

\subsubsection{Definition of Cyclic Groups}
\begin{dft}
  A group $G$ is said to be \textbf{cyclic} if there exists $g\in G$ such that every element of $G$ is equal to $g^{n}$ for some integers $n$. In this case, $g$ is a \textbf{generator} of $G$, denoted by $G=\bra{g}$.
\end{dft}\n

Note that a generator of a cyclic group is not unique, in other words, there may exist different elements $g_{1},g_{2}$ such that $G=\bra{g_{1}}=\bra{g_{2}}$.\n

\begin{thm}
  If a group is cyclic, the group is also abelian.\n

  \prf Let $G$ be a cyclic group, then $G=\bra{g}$ for some element $g\in G$ and every element is in the form $g^{n}$ for some $n\in\Z$, then

  $$g^{n_{1}}\cdot g^{n_{2}}=g^{n_{1}+n_{2}}=g^{n_{2}+n_{1}}=g^{n_{2}}\cdot g^{n_{1}}$$
\end{thm}

\subsection{Symmetric Groups}
\subsubsection{Definition of Symmetric Groups}
\begin{dft}
  Let $X$ be a set. A permutation of the set $X$ is the bijective map

  $$\sigma:X\to X$$
\end{dft}\n

\begin{thm}
  The set $S$ of permutations of a set $X$ is a group with respect to the composition of maps ($\circ$).\n

  \prf The following conditions of a group is satisfied:

  \begin{alist}
    \item Let $\sigma,\gamma$ be permutations of $X$. By definition, they are both bijective maps from $X$ to itself. Then $\sigma\circ\gamma$ will also be a bijective map from $X$ to itslef, hence $\sigma\circ\gamma$ is a permutation of $X$, and $\circ$ is a well-defined binary opeation on $S$.
    \item Composition of maps is associative.
    \item There exists a map $e(x)=x$ for all $x\in X$, which is an identity element in $S$.
    \item Let $\sigma$ be an element in $S$. Since $\sigma$ is bijective, there exists the inverse $\sigma^{-1}$ which is also bijective.
  \end{alist}
\end{thm}\n

\begin{dft}
  If a set $S_{x}$ of permutations of another set $X$ is a group with respect to the composition of maps, then $S_{x}$ is said to be the \textbf{symmetric group} on $X$.\n

  An \textbf{n-th symmetric group} is the symmetric group on $I_{n}:=\brc{1,2,\cdots,n}$ where $n$ is a positive integer, denoted by $S_{n}$. Its element $\sigma\in S_{n}$, which is a bijective map, can be expressed as follows:

  $$\sigma=\brr{\begin{matrix}
    1 & 2 & \cdots & n\\
    \sigma(1) & \sigma(2) & \cdots & \sigma(n)
  \end{matrix}}$$
\end{dft}

\subsubsection{Cyclic Permutations}
\begin{dft}
  Let $S$ be an n-th symmetric group, and $\sigma$ be an element in $S$. $(i_{1}i_{2}\cdots i_{k})$ is a \textbf{k-cycle} which denotes the permutation

  $$i_{1}\mapsto i_{2},i_{2}\mapsto i_{3},\cdots,i_{k}\mapsto i_{1}$$\s

  and $j\mapsto j$ for all

  $$j\in\brc{1,2,\cdots,n}\setminus\brc{i_{1},i_{2},\cdots,i_{k}}$$\s

  which is said to be \textbf{fixed} by $\sigma$.
\end{dft}\n

\begin{exm}
  Consider the element $\sigma$ in eighth symmetric group $S_{8}$:

  $$\begin{aligned}[t]
    \sigma&=\brr{\begin{matrix}
      1 & 2 & 3 & 4 & 5 & 6 & 7 & 8\\
      8 & 6 & 4 & 2 & 1 & 3 & 5 & 7
    \end{matrix}}\\
    &=(1\: 8\: 7\: 5)(2\: 6\: 3\: 4)
  \end{aligned}$$\s

  and such factorization is essentially unique.
\end{exm}\n

Note that a 2-cycle is often called a \textbf{transposition} as it switches two elements with each other.\n

\begin{thm}
  Every permutation in any symmetric group is either a cycle or a product of disjoint cycles.
\end{thm}

\subsection{Dihedral Groups}
\subsubsection{Definition of Dihedral Groups}
\begin{dft}
  A \textbf{dihedral group}, denoted by $D_{n}$, is a group containing elements of transformations of $\R^{2}$, which consists of all rotations (denoted by $r$) by fixed angles about the origin and reflections (denoted by $s$) over lines through the origin, of regular polygons with $n$ sides.
\end{dft}

\subsubsection{Dihedral Groups and Transformations of Regular Polygons}
Note that dihedral groups are subgroups of the group $U$ of rigid motions on $\R^{2}$, where

$$U:=\brc{\sigma:\R^{2}\to\R^{2}\srm\sigma(\nrm{\mathbf{v}-\mathbf{w}})=\nrm{\sigma(\mathbf{v})-\sigma(\mathbf{w})}\;\forall\mathbf{v},\mathbf{w}\in\R^{2}}$$ %propdisp

In fact, $\sigma$ represents the distance in $\R^{2}$, and note that

$$\sigma(\mathbf{v})=A\mathbf{v}+\mathbf{b}$$\s

for some orthogonal $A\in M_{2\times 2}(\R)$ and $\mathbf{b}\in\R^{2}$. With this terminology,

$$T=\brc{\sigma\in U\srm\sigma\;\text{fixes }\mathrm{\sigma}\in\R^{2}}$$\s

contains all rotations and reflections in \rdft.\n

Now consider a regular n-gon $\Delta_{n}\subset\R^{2}$ centered at $\mathbf{O}\in\R^{2}$ with vertices $P_{n}:=\brc{x_{1},x_{2},\cdots,x_{n}}\subset\R^{2}$, then the dihedral group can be defined as

$$D_{n}:=\brc{\sigma\in T\srm\sigma(P_{n})=P(n)}$$

\subsubsection{Properties of Dihedral Groups}
\begin{thm}
  A dihedral group $D_{n}$ can be expressed as

  $$D_{n}=\brc{\begin{matrix}
    r_{0},r_{1},r_{2},\cdots,r_{n-1}\\
    s_{1},s_{2},s_{3},\cdots,s_{n}
  \end{matrix}}$$\s

  where $r_{k}$ is a counterclockwise rotation by $\frac{2k\pi}{n}$ and $s_{1},s_{2},\cdots,s_{n}$ are reflections about the $n$ symmetry axes of $\Delta_{n}$. Then $\abs{D_{n}}=2n$.
\end{thm}\n

Note that the rotations can be rewritten as $\brc{id,r,r^{2},\cdots,r^{n-1}}$ and reflections can be rewritten as $\brc{s,rs,r^{2}s,\cdots,r^{n-1}s}$, since the composition of any two reflections is a rotation.\n

\begin{pst}
  For any rotation $r$ and reflection $s$, $s^{-1}rs=r^{-1}$, then all dihedral groups $D_{n}$ are nonabelian since $rs=sr^{-1}$.
\end{pst}

\subsection{Subgroups}
\subsubsection{Definition of Subgroups}
Just like subsets as substructure of sets in set theory, there is also substructure of groups. The following are the definition of subgroups:\n

\begin{dft}
  Let $(G,*)$ be a group. A nonempty subset $H\subset G$ is a \textbf{subgroup} of $G$, denoted by $H<G$, if $H$ is also a group by itself under the induced structure of $G$. By induced structure, more precisely it means that:

  \begin{alist}
    \item $H$ is closed under $*$, which is $a*b\in H\;\forall a,b\in H$ (This shows that the restriction of $*$ to $H\times H$ gives a well-defined binary operation on $H$).
    \item $H$ is a group using this induced operation.
  \end{alist}
\end{dft}\n

\begin{dft}
  For any group $G$, the singleton containing only the identity element $\brc{e}$ is called the \textbf{trivial subgroup} of $G$.\n

  A subgroup $H<G$ is said to be \textbf{nontrivial} if $H\neq\brc{e}$. Since a group is a subgroup of itself, which is $G<G$, a subgroup $H<G$ is said to be \textbf{proper} if $H\neq G$.
\end{dft}

\begin{exm}
  Below are various examples of subgroups:

  \begin{alist}
    \item Since $\Z\subset\Q\subset\R\subset\C$, then $(\Z,+)<(\Q,+)<(\R,+)<(\C,+)$.
    \item The roots of unity $\mathrm{U}_{n}=\brc{z\in\C\srm z^{n}=1}$ is a subgroup of $\mathrm{U}=\brc{z\in\C\srm\abs{z}=1}$, for all positive integers $n$.
    \item The special linear group $\mathrm{SL}(n,F)=\brc{A\in\mathrm{GL}(n,F)\srm\det(A)=1}$ is a subgroup of general linear group $\mathrm{GL}(n,F)$.
  \end{alist}
\end{exm}

\subsubsection{Subgroups and Symmetric Groups}
\begin{pst}
  Let an n-th symmetric group be $S_{n}$, then each permutation $\sigma\in S_{n}$ is a product of transpositions.\n

  \prf Notice that every cycle is a product of transpositions:

  $$(i_{1}\: i_{2}\:\cdots\: i_{k})=(i_{1}\: i_{k})(i_{1}\: i_{k-1})\cdots(i_{1}\: i_{2})$$\s

  then the result follows from the fact that every permutation of the symmetric group is a product of cycles.
\end{pst}\n

\begin{pst}
  Let an n-th symmetric group be $S_{n}$, then in every factorization of $\sigma\in S_{n}$ as a product of transpositions, then the number of factors is either always even or always odd.\n

  \prf The use of determinant is required to finish this proof. Observe that there exists an unique matrix $A\in M_{n\times n}(\R)$ with only $0$ or $1$ as entries which sends any vector

  $$\brs{\begin{matrix}
    x_{1}\\
    x_{2}\\
    \vdots\\
    x_{n}\\
  \end{matrix}}\in\R^{n}\;\text{to}\brs{\begin{matrix}
    x_{\sigma_{1}}\\
    x_{\sigma_{2}}\\
    \vdots\\
    x_{\sigma_{n}}\\
  \end{matrix}}\in\R^{n}$$\s

  where $A$ is given by permuting rows in the identity matrix $I_{n}$. Now if there are two factorizations of $\sigma$ into products of transpositions, which is

  $$\begin{cases}
    \sigma=\tau_{1}\tau_{2}\cdots\tau_{k}\\
    \sigma=\mu_{1}\mu_{2}\cdots\mu_{l}
  \end{cases}$$\s

  which can be determined that $\det(A)=(-1)^{k}\det(I_{n})=(-1)^{k}$ in the first equation and $\det(A)=(-1)^{l}\det(I_{n})=(-1)^{l}$. Hence $(-1)^{k}=(-1)^{l}$, and $k-l$ is divisible by $2$.
\end{pst}\n

\begin{dft}
  Let an n-th symmetric group be $S_{n}$, then a permutation $\sigma\in S_{n}$ is called \textbf{even} (\textbf{odd}) if it is a product of an even (odd) number of transpositions.\n

  Further let $A_{n}:=\brc{\sigma\in S_{n}\srm\sigma\;\text{is even}}$, then $A_{n}<S_{n}$. $A_{n}$ is called the \textbf{n-th alternating group}.
\end{dft}

\subsubsection{Techniques on Subgroups}
\begin{pst}
  A nonempty subset $H$ of a group $G$ is a subgroup of $G$ if and only if

  $$ab^{-1}\in H\;\forall a,b\in H$$\s

  \prf \prt[$\Rightarrow$]{zr} Suppose $H<G$, then given $a,b\in H$, we have $b^{-1}\in H$ because of the existence of inverse in groups. Along with the closedness of $H$ under the group product, $ab^{-1}\in H$.\n

  \prtc[$\Leftarrow$]{zr} Suppose $ab^{-1}\in H\;\forall a,b\in H$. Because $H$ is nonempty, there exists some element within $H$, and for any element $a\in H$, $e=a\cdot a^{-1}\in H$. For any $b\in H$, $b^{-1}=e\cdot b^{-1}\in H$. Now, for $a,b\in H$, since $b^{-1}\in H$, then $a\cdot b=a(b^{-1})^{-1}\in H$, $H$ is closed under the group opeation. Finally, with the associativity of the induced operation follows from that in $G$, it can be concluded that $H$ is a group under the induced operation.
\end{pst}\n

With the proposition above, it is much easier to determine whether a subset of a group is a subgroup or not.

\begin{exm}
  Prove example (b) and (c) in \rexm[\sctd{5}].\n

  \ans \prt[b]{zb} Note that for any $z_{1},z_{2}\in U_{n}$,

  $$\begin{aligned}[t]
    &z_{1}^{n}=z_{2}^{n}=1\\
    \Rightarrow&(z_{1}z_{2}^{-1})^{n}=1\\
    \Rightarrow&=z_{1}z_{2}^{-1}\in U_{n}
  \end{aligned}$$\s

  By \rpst[\sctd{1}], $U_{n}<U$.\n

  \prtc[c]{zb} Note that for any $A,B\in\mathrm{SL}(n,F)$,

  $$\begin{aligned}[t]
    &\det(A)=\det(B)=1\\
    \Rightarrow&\det(AB^{-1})=\det(A)\det(B)^{-1}=1\\
    \Rightarrow&AB^{-1}\in\mathrm{SL}(n,F)
  \end{aligned}$$\s

  By \rpst[\sctd{1}], $\mathrm{SL}(n,F)<\mathrm{GL}(n,F)$.
\end{exm}

\subsubsection{Cyclic Subgroups}
\begin{pst}
  Let $G$ be a group and fix an element $g\in G$ where $\bra{g}:=\brc{g^{k}\srm k\in\Z}$. $\bra{g}$ is the smallest subgroup of $G$ containing $g$.\n

  \prf For $\bra{g}<G$, take $g^{k_{1}},g^{k_{2}}\in\bra{g}$, then $g^{k_{1}}\cdot (g^{k_{2}})^{-1}=g^{k_{1}-k_{2}}\in\bra{g}$, which means that $\bra{g}<G$. Now let $H<G$ be a subgroup of $G$ containing $g$, then $g^{k}\in H\;\forall k\in\Z$. Therefore, $\bra{g}\subset H$.
\end{pst}\n

With the proposition above, the following definition of cyclic subgroups can be introduced:\n

\begin{dft}
  Let $G$ be a group and fix an element $g\in G$. $\bra{g}$ is the \textbf{cyclic subgroup} generated by $g$.
\end{dft}\n

\begin{pst}
  The intersection of any collection of subgroups of a group $G$ is a subgroup of $G$, that is if $\brc{H_{i}\srm i\in I}$ is a collection of subgroups of $G$, then

  $$\bigcap_{i\in I}H_{i}<G$$
\end{pst}\n

With the proposition above, it can be rewritten into

$$\bra{g}=\bigcap_{\brc{H\srm g\in H<G}}H$$\s

\begin{pst}
  Every subgroup of a cyclic group is cyclic.\n

  \prf Let $G=\bra{g}$ be a cyclic group, and $H<G$ be a subgroup. If $H$ is trivial, then it is cyclic since it is generated by the identity element $e$. If $H$ is nontrivial, then there exists $k\in\Z^{+}$ such that $g^{k}\in H$. Let

  $$m:=\min\brc{k\in\Z^{+}\srm g^{k}\in H}$$\s

  Now claim that $H$ is generated by $g^{m}$. First of all, it is obvious that $\bra{g^{m}}\subset H$. Conversely, let $g^{n}$ be an arbitrary element in $H$. By the Division Algorithm, there exists unique integers $q$ and $0\leq r\leq m-1$ such that $n=mq+r$. Then $g^{n}=(g^{m})^{q}g^{r}$ which leads to $g^{r}=(g^{m})^{-q}g^{n}\in H$. If $r\neq 0$, then it contradicts to the Division Algorithm, so it forces $r=0$. With $r=0$, $g^{n}\in\bra{g^{m}}$ and $H\subset\bra{g^{m}}$.
\end{pst}\n

\begin{crl}
  Any subgroup of $(\Z,+)$ is of the form $n\Z$ for some $n\in\Z$.
\end{crl}

\subsubsection{Greatest Common Divisor}
With the corollary of \rpst[\sct], the greatest common divisor can be defined:\n

\begin{dft}
  For any $a,b\in\Z$, the subset

  $$\bra{a,b}:=\brc{ma+nb\srm m.n\in\Z}$$\s

  is a subgroup of $\Z$, and is of the form $d\Z$ for some $d\in\Z$. The \textbf{greatest common divisor} of $a$ and $b$, defined by $\gcd(a,b)$, is the positive integer $d$ mentioned above. Furthermore, $d$ has to satisfies the following:

  \begin{alist}
    \item \textbf{Definition}

    $$d=ma+nb\;\exists m,n\in\Z$$

    \item \textbf{Divisibility}

    $$d\mid a\;\text{and }d\mid b$$

    \item \textbf{Common divisor}

    $$\text{If }k\mid a\;\text{and }k\mid b\;\text{, then }k\mid d$$
  \end{alist}
\end{dft}\n

\begin{pst}
  Let $G$ be a cyclic group of order $n$ and $g\in G$ be a generator of $G$, which is $G=\bra{g}$. Let $g^{s}\in G$ be an element in $G$, then the order of $g^{s}$

  $$\abs{g^{s}}=\frac{n}{\gcd(s,n)}$$\s

  Moreover, $\bra{g^{s}}=\bra{g^{t}}$ if and only if $\gcd(s,n)=\gcd(t,n)$.\n

  \prf Let $m=\abs{g^{s}}$. Since $\abs{G}=n$, $(g^{s})^{\frac{n}{d}}=(g^{n})^{\frac{s}{d}}=e$. By \rpst[\sctd{26}], $m\mid\frac{n}{d}$. On the other hand, $e=(g^{s})^{m}$ and by the same proposition as above, $n\mid sm$. By dividing both sides by $d$, $\frac{n}{d}\mid\frac{sm}{d}$. However, $\frac{n}{d}$ and $\frac{s}{d}$ are relatively primes, so $\frac{n}{d}\mid{m}$ which proves that $\abs{g^{s}}=\frac{n}{\gcd(s,n)}$.\n

  To prove the second assertion, it has to be first shown that there is an equality of subgroups $\bra{g^{s}}=\bra{g^{d}}$ where $d=\gcd(s,n)$. Note that one inclusion is clear: as $d\mid s$, $g^{s}\in \bra{g^{d}}$ which implies $\bra{g^{s}}\subset\bra{g^{d}}$. Conversely, note that there exists $p,q\in\Z$ such that $d=ps+qn$, so $g^{d}=(g^{s})^{p}(g^{n})^{q}=(g^{s})^{p}=\bra{g^{s}}$. Hence $\bra{g^{d}}\subset\bra{g^{s}}$ and the proof of equality is complete.\n

  $\bra{g^{s}}=\bra{g^{t}}$ implies that $\abs{g^{s}}=\abs{g^{t}}$ which in turn gives $\gcd(s,n)=\gcd{t,n}$. Conversely, if $d:=\gcd(s,n)=\gcd(t,n)$, then $\bra{g^{d}}=\bra{g^{s}}=\bra{g^{t}}$.
\end{pst}\n

\begin{crl}
  All generators of a cyclic group $G=\bra{g}$ of order $n$ are of the form $g^{r}$ where $r$ and $n$ are relatively primes.
\end{crl}

\subsection{Generating Sets}
\subsubsection{Definition of Generating Sets}
\begin{dft}
  Let $G$ be a group and $S$ be a nonempty subset of $G$. Define

  $$\bra{S}:=\brc{a_{1}^{m_{1}}a_{2}^{m_{2}}\cdots a_{n}^{m_{n}}\srm n\in\N,a_{i}\in S\;\forall i,m_{i}\in\Z\;\forall i}$$\s

  be the \textbf{smallest subgroup} of $G$ containing the subset $S$, which is called the subgroup generated by $S$.\n

  \prf[zg] Let $a_{1}^{m_{1}}a_{2}^{m_{2}}\cdots a_{n}^{m_{n}},b_{1}^{l_{1}}b_{2}^{l_{2}}\cdots b_{k}^{l_{k}}\in\bra{S}$. Then $AB^{-1}=a_{1}^{m_{1}}a_{2}^{m_{2}}\cdots a_{n}^{m_{n}}b_{1}^{-l_{1}}b_{2}^{-l_{2}}\cdots b_{k}^{-l_{k}}\in\bra{S}$. Then $\bra{S}<G$.\n

  Furthermore, if $H<G$ is a subgroup containing $S$, then $a^{k}\in H\;\forall a\in S,\forall k\in\Z$ and thus $a_{1}^{m_{1}}a_{2}^{m_{2}}\cdots a_{n}^{m_{n}}\in H\;\forall a_{1},a_{2},\cdots,a_{n}\in S,\forall k_{1},k_{2},\cdots,k_{n}\in\Z$. Therefore, $\bra{S}\subset H$.
\end{dft}\n

Note that the subgroup generated by an empty set $\bra{\phi}=\brc{e}$. Also note that

$$\bra{S}=\bigcap_{S\subset H<G}H$$\s

If $S=\brc{a_{1},a_{2},\cdots,a_{n}}$, it is often written as $\bra{a_{1},a_{2},\cdots,a_{n}}$ instead of $\bra{\brc{a_{1},a_{2},\cdots,a_{n}}}$.\n

\begin{dft}
  A group $G$ is said to be \textbf{finitely generated} if there exists a finite subset $S=\brc{a_{1},a_{2},\cdots,a_{n}}$ such that $G=\bra{a_{1},a_{2},\cdots,a_{n}}$.
\end{dft}

\subsubsection{Equivalence Relations and Partitions}
Recall the definition of partitions:\n

\begin{dft}
  Let $S$ be a set. A \textbf{partition} of $S$, denoted by $P$, is a collection of subsets $\brc{S_{i}\srm i\in I}$ such that $S_{i}\neq\phi\;\forall i\in I, S_{i}\cap S_{j}=\phi\;\text{if }i\neq j,S=\bigcap_{i\in I}S_{i}$. In this case, it is also said that $S$ is a disjoin union of $\brc{S_{i}\srm i\in I}$ written as $S=\bigsqcup_{i\in I}S_{i}$.
\end{dft}\n

\begin{dft}
  Let $S$ be a set. An \textbf{equivalence relation} on $S$, is a relaion $\sim$ on $S$ (which is a subset of $S\times S$) such that the following are true:

  \begin{alist}
    \propdisp

    \item \textbf{Reflexive}

    $$a\sim a\;\forall a\in S$$

    \item \textbf{Symmetric}

    $$\text{If }a\sim b\;\text{then }b\sim a$$

    \item \textbf{Transitive}

    $$\text{If }a\sim b\;\text{and }b\sim c\;\text{then }a\sim c$$
  \end{alist}
\end{dft}\n

\begin{dft}
  For any $a\in S$, the subset

  $$C_{a}:=\brc{b\in S\srm b\sim a}\subset S$$\s

  is called the \textbf{equivalence class} of $a$.
\end{dft}\n

\begin{pst}
  The collection of equivalence classes $\brc{C_{a}\srm a\in S}$ is a partition of $S$. More precisely, any two equivalence classes are either exactly the same subset or disjoint, which is

  $$C_{a}\cap C_{b}\neq\phi\Rightarrow C_{a}=C_{b}$$\s

  Then

  $$S=\bigcup_{a\in S}C_{a}=\bigsqcup_{a\in I}C_{a}$$\s

  for some index set $I$.\n

  \prf Suppose $c\in C_{a}\cap C_{b}\neq\phi$, then

  $$c\in C_{a}\Rightarrow c\sim a\Rightarrow a\sim c$$\s

  On the other hand,

  $$c\in C_{b}\Rightarrow c\sim b\Rightarrow a\sim b$$\s

  In order to show that $C_{a}\subset C_{b}$, let $d\in C_{a}$ where $d\sim a$ is always true. However $a\sim b$, and by transitivity, $d\sim b$ and so $d\in C_{b}$. By the symmetric property, $a\sim b\Rightarrow b\sim a$. Also note that $C_{a}$ and $C_{b}$ are interchangable, then $C_{b}\subset C_{a}$. Hence $C_{a}=C_{b}$.
\end{pst}

\subsubsection{Application of Equivalence Relations}
Recall \rthm[\sctd{20}], below is the proof of the theorem:\n

\begin{prv}
  Let $\sigma\in S_{n}$ be a permutation of $I_{n}=\brc{1,2,\cdots,n}$. Define a relation $\sim$ on $I_{n}$ by $a\sim b$ if and only if $b=\sigma^{k}(a)$ for some $k\in\Z$, which is an equivalence relation.\n

  Hence it induces a partition of $I_{n}$:

  $$I_{n}=O_{1}\sqcup O_{2}\sqcup\cdots\sqcup O_{m}$$\s

  where $O_{i}$ is called an \textbf{orbit} of $\sigma$ in $I_{n}$. Then for $j=1,2,\cdots,m$, define a cycle of length $\abs{O_{j}}$ by

  $$\mu_{j}(a)=\begin{cases}
    \sigma(a)&\erm{, if }a\in O_{j}\\
    a&\erm{, if }a\not\in O_{j}
  \end{cases}$$\s

  Note that $\mu_{1},\mu_{2},\cdots,\mu_{m}$ are disjoint cycles because $\brc{O_{1},O_{2},\cdots,O_{m}}$ is a partition of $I_{n}$.\n

  Finally, check that $\sigma=\mu_{1}\mu_{2}\cdots\mu_{m}$. By computing

  $$\begin{aligned}[t]
    (\mu_{1}\mu_{2}\cdots\mu_{m})(a)&=\mu_{j}(a)\erm{where }a\in O_{j}\\
    &=\sigma(a)
  \end{aligned}$$
\end{prv}

\subsection{Cosets}
\subsubsection{Definition of Cosets}
\begin{pst}
  Let $G$ be a group and $H<G$ be a subgroup, then a relation $\sim_{L}$ on $G$ can be defined as $a\sim_{L}b$ if and only if $b=a\cdot h$ for some $h\in H$, or in other words, $a^{-1}b\in H$, then $\sim_{L}$ is an equivalence relation.\n

  \prf In order to prove $\sim_{L}$ is an equivalence relation, check the following:

  \begin{alist}
    \item \textbf{Reflexive}

    $$\begin{aligned}[t]
      &a^{-1}a=e\in H\;\forall a\in G\\
      \Rightarrow&a\sim_{L}a\;\forall a\in G
    \end{aligned}$$

    \propdisp

    \item \textbf{Symmetric}

    $$\begin{aligned}[t]
      a\sim_{L}b&\Leftrightarrow a^{-1}b\in H\\
      &\Rightarrow (a^{-1}b)^{-1}=b^{-1}a\in H\\
      &\Rightarrow b\sim_{L}a
    \end{aligned}$$

    \item \textbf{Transitive}

    $$\begin{aligned}[t]
      \begin{cases}
        a\sim_{L}b\\
        b\sim_{L}c
      \end{cases}&\Leftrightarrow\begin{cases}
        a^{-1}b\in H\\
        b^{-1}c\in H
      \end{cases}\\
      &\Rightarrow a^{-1}c=(a^{-1}b)(b^{-1}c)\in H\\
      &\Rightarrow a\sim_{L}c
    \end{aligned}$$

  \end{alist}
\end{pst}\n

Note that for any subset $H\subset G$, the relation $\sim$ on $G$ defined by $a\sim b$ if and only if $a^{-1}b\in H$ is an equivalence relation, if and only if $H<G$.

\begin{dft}
  Let $G$ be a group and $H<G$ be a subgroup. Define $\sim_{L}$ be the relation mentioned above, then $\sim_{L}$ induces a partition of $G$ into equivalence classes, where those classes are \textbf{left cosets} of $H$ in $G$. Each left coset of $H$ in $G$ is of the form

  $$\begin{aligned}[t]
    aH&=\brc{b\in G\srm a\sim_{L}b}\\
    &=\brc{b\in G\srm b=ah\;\text{for some }h\in H}
  \end{aligned}$$\s

  Similarly, define $\sim_{R}$ be another relation where $a\sim_{R}b\Leftrightarrow b=ha$ for some $h\in H$, then $\sim_{R}$ also induces a partition of $G$ into equivalence classes, where those classes are \textbf{right cosets} of $H$ in $G$. Each right coset of $H$ in $G$ is of the form

  $$\begin{aligned}[t]
    Ha&=\brc{b\in G\srm a\sim_{R}b}\\
    &=\brc{b\in G\srm b=ha\;\text{for some }h\in H}
  \end{aligned}$$
\end{dft}\n

Note that left cosets and right cosets are equivalent if the group is abelian.

\subsubsection{Theorem of Lagrange}
With the definition of left cosets and right cosets, it can be related with the following:\n

\begin{dft}
  The \textbf{index} of $H$ in $G$ is the number of left cosets (or right cosets) of $H$ in $G$, denoted by $[G:H]$.
\end{dft}\n

The definition above implies that the number, or cardinality to be exact, of left cosets and right cosets of $H$ in $G$ are equal. This leads to the \textbf{Theorem of Lagrange}:\n

\begin{thm}
  Let $H$ be a subgroup of a group $G$, then for all $a\in G$,

  $$\abs{aH}=\abs{H}=\abs{Ha}$$\s

  In particular, if $\abs{G}<+\infty$, then $\abs{H}\mid\abs{G}$ or more precisely,

  $$\abs{G}=\abs{H}\cdot[G:H]$$\s

  \prf For the first statement, consider the maps $\varphi_{a}:H\to aH$ and $\psi_{a}:H\to Ha$. In other words, for any $h\in H$, $\varphi_{a}:h\mapsto ah$ and $\psi_{a}:h\mapsto ha$. Claim $\varphi_{a}$ (and $\psi_{a}$) is a bijection for all $a\in G$:

  \begin{alist}
    \item \textbf{Injectivity}\\
    For all $h_{1},h_{2}\in H$, $\varphi_{a}(h_{1})=\varphi_{a}(h_{2})\Rightarrow ah_{1}=ah_{2}\Rightarrow h_{1}=h_{2}$. Hence $\varphi_{a}$ is injective.
    \item \textbf{Surjectivity}\\
    Let $g\in aH$, then by definition, there exists $h\in H$ such that $g=ah=\varphi_{a}(h)$. Hence $\varphi_{a}$ is surjective.
  \end{alist}

  For the second statement, note that the partition of $G$

  $$G=\bigsqcup_{i\in I}a_{i}H$$\s

  where $\brc{a_{1}H,a_{2}H,\cdots,a_{k}H}$ enumerates the set of all left cosets of $H$ in $G$. This implies that

  $$\begin{aligned}[t]
    \abs{G}&=\sum_{i=1}^{k}\abs{a_{i}H}\\
    &=\sum_{i=1}^{k}\abs{H}\\
    &=k\abs{H}=[G:H]\cdot\abs{H}
  \end{aligned}$$
\end{thm}\n

Below are some corollaries based on the Theorem of Lagrange:\n

\begin{crl}
  Let $G$ be a finite group, then $\abs{g}\mid\abs{G}$ for all $g\in G$. In particular,

  $$g^{\abs{G}}=e\;\forall g\in G$$\s

  \prf By applying the Theorem of Lagrange to the cyclic subgroup $\bra{g}$ generated by $g\in G$, $\abs{g}=\abs{\bra{g}}\mid\abs{G}$.
\end{crl}\n

\begin{crl}
  If $G$ is a finite group of prime order, then it is cyclic.\n

  \prf Suppose $\abs{G}=p$ is a prime. Further let $g\in G\setminus\brc{e}$, then $\abs{g}\mid\abs{G}=p$ by the corollary above, and $\abs{g}=p$. Therefore $G=\bra{g}$.
\end{crl}\n

\begin{exm}
  If $\brc{e}<G$, then the set of left cosets and the set of right cosets are the same, which is $\brc{\brc{g}\srm g\in G}$. This means that $G=\bigsqcup_{g\in G}\brc{g}$.\n

  On the other hand, if $G<G$, then again the set of left cosets and the set of right cosets are the same, which is $\brc{G}$. This leads to a trivial result $G=G$.
\end{exm}\n

\begin{exm}
  Let $D_{n}$ be a dihedral group, which is

  $$D_{n}=\brc{\begin{matrix}
    id,r_{1},r_{2},\cdots,r_{n-1}\\
    s,r_{1}s,r_{2}s,\cdots,r_{n-1}s
  \end{matrix}}$$\s

  For $\bra{r}<D_{n}$,

  $$\begin{aligned}[t]
    \text{left cosets}&=\brc{\bra{r},s\bra{r}}\Rightarrow D_{n}=\bra{r}\sqcup s\bra{r}\\
    \text{right cosets}&=\brc{\bra{r},\bra{r}s}\Rightarrow D_{n}=\bra{r}\sqcup \bra{r}s
  \end{aligned}$$\s

  In particular, $s\bra{r}=\bra{r}s$.
\end{exm}

\subsection{Group Homomorphisms}
\subsubsection{Definition of Group Homomorphisms}
\begin{dft}
  Let $G=(G,*)$ and $G'=(G,*')$ be groups. A \textbf{group homomorphism}, denoted by $\phi$, is a map $G\to G'$ such that

  $$\phi(a*b)=\phi(a)*'\phi(b)$$\s

  for all $a,b\in G$. If $\phi$ is an addition bijective, then $\phi$ is called a \textbf{group isomorphism}, and $G$ is \textbf{isomorphic} to $G'$ as groups. In other words, $G$ and $G'$ have the same structure.
\end{dft}\n

\begin{dft}
  Let $G$ be a group. An \textbf{automorphism} of $G$ is an isomorphism from $G$ onto itself

  $$\phi:G\to G$$
\end{dft}\n

Note that

$$\textrm{Aut}(G):=\brc{\phi:G\to G\srm\phi\;\text{is an isomorphism}}$$\s

is a group itself under composition of maps.\n

\begin{exm}
  Below are various examples of group homomorphisms:

  \begin{alist}
    \item The exponential function

    $$\exp:(\R,+)\to(\R^{+},\cdot)$$\s

    is a group homomorphism, which is $\exp(a+b)=\exp(a)\exp(b)$.
    \item The determinant

    $$\det:\GL(n,\R)\to(\R^{x},\cdot)$$\s

    is a group homomorphism, which is $\det(AB)=\det(A)\det(B)$.
    \item Fix any $n\in\Z^{x}$, then $n\Z<\Z$ and the map

    $$\phi:n\Z\to\Z$$\s

    is a group isomorphism.
  \end{alist}
\end{exm}\n

Note that if $\phi:G\to G'$ is a group isomorphism, then $\phi^{-1}:G'\to G$ is automatically a group isomorphism. For example, the logarithmic function is the inverse of the exponential function in example (a) of \rexm[\sct], and it is an isomorphism, which is $\log(ab)=\log(a)+\log(b)$.\n

Also note that in example (c) of \rexm[\sct], if $\abs{n}\neq 1$, then $n\Z\neq\Z$ and $n\Z<\Z$ but $n\Z\cong\Z$. In general, given a group $G$ and two subgroups $H,H'<G$, then $H=H'$ as subsets in $G$, and $H\cong H'$ as groups, which are usually not equivalent. On the other hand, for any $n\in\Z$, the map $\phi':\Z\to n\Z$ is a group homomorphism but not an isomorphism unless $\abs{n}=1$.\n

\begin{exm}
  The remainder map

  $$\phi:\Z\to\Z_{n}=\brc{0,1,2,\cdots,n-1}$$\s

  is well-defined because of the Division Theorem, and surjective. Let $k_{1},k_{2}\in\Z$, and check that $\phi(k_{1}+k_{2})=\phi(k_{1})+\phi(k_{2})$ for homomorphism. By the Division Theorem, for $i=1,2$, there exists unique $q_{i},r_{i}\in\Z$ with $0\leq r_{i}\leq n-1$ such that $k_{i}=q_{i}n+r_{i}$.\n

  Now by definition, $\phi(k_{i})=r_{i}$ for $i=1,2$. Consider $k_{1}+k_{2}$,

  $$k_{1}+k_{2}=(q_{1}+q_{2})n+(r_{1}+r_{2})=(q_{1}+q_{2}+1)n+(r_{1}+r_{2}-n)$$\s

  which leads to

  $$\begin{aligned}[t]
    \phi(k_{1}+k_{2})&=\begin{cases}
      r_{1}+r_{2}&\erm{if }0\leq r_{1}+r_{2}\leq n-1\\
      r_{1}+r_{2}-n&\erm{if }n\leq r_{1}+r_{2}
    \end{cases}\\
    &=r_{1}+_{n}r_{2}=\phi(k_{1})+_{n}\phi(k_{2})
  \end{aligned}$$
\end{exm}\n

\subsubsection{Basic Properties of Group Homomorphisms}
\begin{pst}
  Let $\phi:G\to G'$ be a group homomorphism, then the following applies:

  \begin{alist}
    \item

    $$\phi(e_{G})=\phi(e_{G'})$$

    \item

    $$\phi(g^{-1})=\phi(g)^{-1}\;\forall g\in G$$

    \item

    $$\phi(g^{n})=\phi(g)^{n}\;\forall g\in G,\forall n\in Z$$
  \end{alist}

  \prf \prt[a]{zr}

  $$\begin{aligned}[t]
    e_{G}\cdot e_{G}&=e_{G}\\
    \phi(e_{G}\cdot e_{G})&=\phi(e_{G})\\
    \phi(e_{G})^{-1}\phi(e_{G})\phi(e_{G})&=\phi(e_{G})^{-1}\phi(e_{G})\\
    \phi(e_{G})=e_{G'}
  \end{aligned}$$\s

  \prtc[b]{zr} For any $g\in G$,

  $$\phi(g^{-1})\phi(g)=\phi(g^{-1}g)=e_{G"}$$\s

  by part (a). Similarly, $\phi(g)\phi(g^{-1})=e_{G'}$, and so $\phi(g^{-1})=\phi(g)^{-1}$ by uniqueness of inverses in $G'$.

  \prtc[c]{zr} Can be proven by using part (a), part (b) and induction.
\end{pst}

\subsubsection{Images and Kernels}
\begin{pst}
  Let $\phi:G\to G'$ be a group homomorphism, then the following applies:

  \begin{alist}
    \item

    $$H<G\Rightarrow\phi(H)<G'$$

    \item

    $$H'<G'\Rightarrow\phi^{-1}(H')<G$$
  \end{alist}

  \prf \prt[a]{zr} Recall the criterion where $\phi\neq H\subset G$ is a subgroup if and only if $ab^{-1}\in H$ for all $a,b\in H$. First, $H\neq\phi$ implies $\phi(H)\neq\phi$. Let $a',b'\in\phi(H)$ be two arbitrary elements. By definition, there exists $a,b\in H$ such that $a'=\phi(a)$ and $b'=\phi(b)$, then

  $$(a')(b')^{-1}=\phi(a)\phi(b)^{-1}=\phi(a)\phi(b^{-1})=\phi(ab^{-1})$$\s

  Since $H<G$ and $ab^{-1}\in H$, $(a')(b')^{-1}\in\phi(H)$, which means that $\phi(H)<G'$.\n

  \prtc[b]{zr} First, $\phi^{-1}(H')\neq\phi$ since $H'\neq\phi$. Let $a,b\in\phi^{-1}(H')$, which is $\phi(a),\phi(b)\in H'$. $H'<G'$ implies that

  $$\phi(a)\phi(b)^{-1}=\phi(ab^{-1})\in H'$$\s

  meaning that $ab^{-1}\in\phi^{-1}(H')$, and so $\phi^{-1}(H')<G$.
\end{pst}\n

\begin{crl}
  Let $\phi:G\to G'$. The \textbf{image} of $\phi$

  $$\mathrm{im}\:\phi:=\phi(G)=\brc{\phi(g)\srm g\in G}$$\s

  is a subgroup of $G'$. The \textbf{kernel} of $\phi$

  $$\ker\phi:=\phi^{-1}(e_{G'})=\brc{g\in G\srm \phi(g)=e_{G'}}$$\s

  is a subgroup of $G$.
\end{crl}

\subsubsection{Applications of Group Homomorphisms}
\begin{pst}
  Let $\phi:G\to G'$ be a surjective group homomorphism, then the following applies:

  \begin{alist}
    \item If $G$ is cyclic, $G'$ is also cyclic.
    \item If $G$ is abelian, $G'$ is also abelian.
  \end{alist}

  \prf \prt[a]{zr} Note that the fact that $\phi$ is injective implies $G'=\phi(G)$. If $G=\bra{g}$ for some $g\in G$, then $G'=\phi(G)=\bra{\phi(G)}$.\n

  \prtc[b]{zr} For $\phi(a),\phi(b)\in G'=\phi(G)$,

  $$\begin{aligned}[t]
    \phi(a)\phi(b)&=\phi(ab)
    &=\phi(ba)=\phi(b)\phi(a)
  \end{aligned}$$\s

  Therefore $G'$ is abelian.
\end{pst}\n

\begin{crl}
  If $G\cong G'$, then $G$ is cyclic (abelian) if and only if $G'$ is cyclic (abelian).
\end{crl}\n

\begin{exm}
  Below are various examples based on \rpst[\sctd{1}]:

  \begin{alist}
    \item $\Z_{4}\not\cong\Z_{2}\times\Z_{2}$ since $\Z_{4}$ is cyclic but $\Z_{2}\times\Z_{2}$ is not. In fact, $\Z_{4}$ and $\Z$ are the only two groups of order $4$ up to isomorphism.
    \item Although $\abs{S_{3}}=\abs{\Z_{6}}=6$, but $S_{3}\not\cong\Z_{6}$ since $\Z_{6}$ is abelian but $S_{3}$ is not. However, $\Z_{6}\cong\Z_{2}\times\Z_{3}$ with $(1,1)$ as generator of $\Z_{2}\times\Z_{3}$.
  \end{alist}
\end{exm}

\subsubsection{Group Homomorphisms and Orders}
\begin{pst}
  If $\phi:G\to G'$ is an isomorphism, then

  $$\abs{\phi(g)}=\abs{g}\;\forall g\in G$$\s

  \prf If $\phi:G\to G'$ is a group homomorphism and $H<G$, then $\phi\!\mid_{H}:H\to G'$ is also a group homomorphism. Also, if $\phi$ is injective, then $\phi\!\mid_{H}$ is also injective.\n

  Apply this to $H=\bra{g}$, which gives an injective homomorphism

  $$\phi\!\mid_{\bra{g}}:\bra{g}\to G'$$\s

  whose image is given by

  $$\mathrm{im}(\phi\!\mid_{\bra{g}})=\brc{\phi(g^{k})=\phi(g)^{k}\srm k\in\Z}=\bra{\phi(g)}$$\s

  Hence $\phi\!\mid_{\bra{g}}$ gives an isomorphism from $\bra{g}$ onto $\bra{\phi(g)}$. In particular, $\abs{g}=\abs{\phi(g)}$.
\end{pst}\n

\begin{exm}
  Consider $\Z_{2}\times\Z_{12}$ and $\Z_{3}\times\Z_{8}$. They are not isomorphic because $\Z_{2}\times\Z_{12}$ has an order $12$ element $(0,1)\in\Z_{2}\times\Z_{12}$, but $\Z_{3}\times\Z_{8}$ does not.
\end{exm}

\propdisp

\subsection{Classification of Groups}
\subsubsection{Classification of Cyclic Groups}
\begin{thm}
  Let $G$ be a cyclic group. If $\abs{G}=+\infty$, then $G\cong\Z$. Otherwise, if $\abs{G}=n<+\infty$, then $G\cong\Z_{n}$.\n

  \prf Let $G=\bra{g}$ for some $g\in G$. The proof is split into two cases:

  \begin{alist}
    \item \textbf{Case I: }$\abs{G}=+\infty$\n

    For $\phi$ to be a homomorphism, note that

    $$\begin{aligned}[t]
      \phi(k_{1}+k_{2})&=g^{k_{1}+k_{2}}=g^{k_{1}}g^{k_{2}}\\
      &=\phi(k_{1})\phi(k_{2})\;\forall k_{1},k_{2}\in\Z
    \end{aligned}$$\s

    For $\phi$ to be injective, if $\phi(k_{1})\neq\phi(k_{2})$ and without the loss of generality, $k_{1}\geq k_{2}$, then

    $$\begin{aligned}[t]
      g^{k_{1}}=g^{k_{2}}&\Rightarrow g^{k_{1}-k_{2}}=e\\
      &\Rightarrow k_{1}-k_{2}=0\erm{since }\abs{g}=+\infty\\
      &\Rightarrow k_{1}=k_{2}
    \end{aligned}$$\s

    Since $G$ is generated by $g$, $\phi$ is surjective. With all the requirements, $\phi$ is an isomorphism.
    \item \textbf{Case II: }$\abs{G}=n<+\infty$\n

    Note that

    $$\begin{cases}
      G=\bra{G}\\
      \abs{G}=n
    \end{cases}\Rightarrow G=\brc{e,g,g^{2},\cdots,g^{n-1}}$$\s

    Then there is a bijection

    $$\phi:G\to\Z_{n}=\brc{0,1,2,\cdots,n-1}$$\s

    Again, for $\phi$ to be a homomorphism, note that

    $$\begin{aligned}[t]
      \phi(g^{k_{1}}+g^{k_{2}})&=\phi(g^{k_{1}+k_{2}})\\
      &=\begin{cases}
        \phi(g^{k_{1}+k_{2}})&\erm{if }k_{1}+k_{2}\leq n-1\\
        \phi(g^{k_{1}+k_{2}-n})&\erm{if }n\leq k_{1}+k_{2}\leq 2n-2
    \end{cases}\\
    &=\begin{cases}
      k_{1}+k_{2}&\erm{if }k_{1}+k_{2}\leq n-1\\
      k_{1}+k_{2}-n&\erm{if }n\leq k_{1}+k_{2}\leq 2n-2
    \end{cases}\\
    &=\phi(g^{k_{1}})+\phi(g^{k_{2}})
    \end{aligned}$$\n

    With all the requirements, $\phi$ is an isomorphism.
  \end{alist}
\end{thm}\n

Note that for any fixed order, there is a unique cyclic group up to isomorphisms.\n

\begin{crl}
  Let $G$ be a cyclic group. If $\abs{G}=p$ where $p$ is a prime number, then $G\cong\Z_{p}$.\n

  \prf By the corollary of the Theorem of Lagrange, if $\abs{G}=p$ is prime, then $G$ is cyclic. The result follows from this, together with the second statement above.
\end{crl}\n

\begin{exm}
  For any positive integer $n$, the subgroup

  $$U_{n}=\brc{z\in\C\srm z^{n}=1}\subset U=\brc{z\in\C\srm \abs{z}=1}=\bra{e^{\frac{2\pi i}{n}}}$$\s

  Since $\abs{U_{n}}=n$, by \rthm[\sctd{1}], $U_{n}\cong\Z_{n}$, and an isomorphism is given by $e^{\frac{2\pi i}{n}}\mapsto k$ for $k=0,1,2,\cdots,n-1$.
\end{exm}

\pagebreak

\section{Ring Theory and General Field Theory}
\subsection{Rings}
\subsubsection{Definition of Rings}
\begin{dft}
  A \textbf{ring} $(R,+,*)$ is a set $R$ equipped with two binary operations

  $$+,*:R\times R\to R$$\s

  such that the following conditions are satisfied:

  \begin{alist}
    \item $(R,+)$ is an abelian group.
    \item $(R,*)$ satisfies the following conditions:

    \begin{rlist}
      \item The multiplication $*$ is associative, which is

      $$(a*b)*c=a*(b*c)\erm{for all }a,b,c\in R$$

      \item There exists an element $1\in R$ (which is called the \textbf{multiplicative identity}) such that

      $$1*a=a*1=a\erm{for all }a\in R$$
    \end{rlist}
    \item $(R,+,*)$ satisfies the distributive laws:

    \begin{rlist}
      \item

      $$a*(b+c)=a*b+a*c\erm{for all }a,b,c\in R$$

      \item

      $$(a+b)*c=a*c+b*c\erm{for all }a,b,c\in R$$
    \end{rlist}
  \end{alist}
\end{dft}\n

Note that in the literature, $(R,+,*)$ without part (b)(ii) is also called a ring. In that case, if part (b)(ii) is also satisfied, then $(R,+,*)$ is called a \textbf{ring with unity}. For example, $(2\Z,+,*)$ is a ring without unity.\n

\begin{exm}
  The following sets equipped with their corresponding usual operations of addition and multiplication are rings:

  \begin{alist}
    \item $\Z,\Q,\R,\C$.
    \item $\Z[x],\Q[x],\R[x],\C[x]$, which represents the set of all polynomials with integer, rational, real and complex coefficients respectively.
    \item $\Q[\sqrt{2}]=\brc{a+b\sqrt{2}\srm a,b\in\Q}\subset\R$.
    \item $\Z[i]=\brc{a+bi\srm a,b\in\Z}\subset\C$. Such ring is called the \textbf{ring of Gaussian integers}.
    \item $\Q[\alpha]:=\brc{f(\alpha)\srm f(x)\in\Q[x]}\subset\C,\Z[\alpha]:=\brc{g(\alpha)\srm g(x)\in\Z[x]}\subset\C$ for any $\alpha\in\C$.
    \item $M_{n\times n}(\R)$ for any fixed positive integer $n$ with usual matrix addition and multiplication.
    \item $C[a,b]=\brc{f:[a,b]\to\R\srm f\;\text{is continuous}},D[a,b]=\brc{g:[a,b]\to\R\srm g\;\text{is differentiable}}$.
    \item $\text{Map}(X,R):=\brc{f:X\to R}$ for any set $X$ and ring $R$.
  \end{alist}
\end{exm}

\subsubsection{Properties of Rings}
\begin{pst}
  Let $R$ be a ring, then $R$ has a unique additive identity and also a unique multiplicative identity.\n

  \prf The additive identity is automatically unique by the definition of rings. The uniqueness of $0$ is also given. Now suppose there are two multiplicative identities $1,1'\in R$, then $1=1\cdot 1'=1'$.
\end{pst}\n

\begin{pst}
  Let $R$ be a ring. For any $r\in R$, its \textbf{additive inverse} $-r\in R$ is unique. If $r\in R$ has a \textbf{multiplicative inverse}, denoted by $r'\in R$ such that $r\cdot r'=r'\cdot r=1$, then the multiplicative inverse is unique.\n

  \prf The part about additive inverse is automically true by the definition of rings. For multiplicative inverses, if $r',r''\in R$ are multiplicative inverses to $r\in R$, then

  $$r'=r'\cdot 1=r'\cdot(r\cdot r'')=(r'\cdot r)\cdot r''=r''$$
\end{pst}\n

\begin{pst}
  Let $R$ be a ring, then the following equations can be applied:

  \begin{alist}
    \item

    $$0\cdot r=r\cdot 0=0\;\forall r\in R$$

    \item

    $$(-1)(-r)=(-r)(-1)=r\;\forall r\in R$$

    \item

    $$(-1)r=r(-1)=-r\;\forall r\in R$$
  \end{alist}

  \prf \prt[a]{zr}

  $$0\cdot r=(0+0)\cdot r=0\cdot r+0\cdot r\Rightarrow 0\cdot r=0$$\s

  Similarly, $r\cdot 0=0$.\n

  \propdisp

  \prtc[b]{zr}

  $$(-1)(-r)+(-r)=((-1)+1)(-r)=0(-r)=0$$\s

  thus uniqueness of additive inverse $-r$ implies that $(-1)(-r)=r$. Similarly, $(-r)(-1)=r$.
\end{pst}\n

\begin{pst}
  If $R$ is a ring where the multiplicative identity is the additive identity, which is $1=0$, then $R=\brc{0}$ which is called the \textbf{zero ring}.\n

  \prf For any $r\in R$, $r=r\cdot 1=r\cdot 0=0$.
\end{pst}

\subsubsection{Commutative Rings}
\begin{dft}
  A ring $R$ is said to be \textbf{commutative} if

  $$ab=ba\;\forall a,b\in R$$
\end{dft}\n

\begin{exm}
  The following sets equipped with their corresponding usual operations of addition and multiplication are commutative rings:

  \begin{alist}
    \item $\Z,\Q,\R,\C$.
    \item $\Z[x],\Q[x],\R[x],\C[x]$.
    \item $\Z[i],\Q[\sqrt{2}],C[a,b],D[a,b]$.
    \item $\mathrm{Map}(X,R)$ where $X$ is a set and $R$ is a commutative ring.
  \end{alist}
\end{exm}

\subsection{Common Types of Rings}
\subsubsection{Modulo Arithmetic}
Recall how the remainder is defined in groups:\n

\begin{dft}
  Let $n$ be a positive integer. Consider the group

  $$\brr{\Z_{m}=\brc{0,1,2,\cdots,m-1},+_{m}}$$\s

  For any $k\in\Z$, the \textbf{remainder} of $k$ divided by $m$, denoted by $\overline{k}$, is equal to $r\in\Z_{m}$ where $k=mq+r$.
\end{dft}\n

Now define an operation of multiplication $\cdot_{m}$ on $\Z_{m}$ by $a\cdot_{m}b=\overline{a\cdot b}$. Note that $+_{m}$ and $\cdot_{m}$ are addition and multiplication defined modulo $m$. Note that

$$\begin{aligned}[t]
  &a\equiv c(\mathrm{mod}\; m)\;\text{and }b\equiv d(\mathrm{mod}\; m)\\
  \Rightarrow&a+b\equiv c+d(\mathrm{mod}\; m)\;\text{and }a\cdot b\equiv c\cdot d(\mathrm{mod}\; m)
\end{aligned}$$\s

the the following proposition can be applied:\n

\begin{pst}
  $(Z_{m},+_{m},\cdot_{m})$ is a commutative ring.\n

  \prf Note that $(Z_{m},+_{m})$ is an abelian group. Let $a,b,c\in\Z_{m}$, then

  $$\begin{aligned}[t]
    a\cdot_{m}(b\cdot_{m}c)&=a\cdot_{m}\overline{bc}=\overline{a\cdot\overline{bc}}\\
    &=\overline{\overline{a}\cdot\overline{bc}}=\overline{abc}
  \end{aligned}$$\s

  and

  $$\begin{aligned}[t]
    (a\cdot_{m}b)\cdot_{m}c&=\overline{ab}\cdot_{m}c=\overline{\overline{ab}\cdot c}\\
    &=\overline{\overline{ab}\cdot\overline{c}}=\overline{abc}
  \end{aligned}$$\s

  Therefore $\cdot_{m}$ is associative. With $1\in\Z_{m}$ as the multiplicative identity, then

  $$\begin{aligned}[t]
    a\cdot_{m}(b+_{m}c)&=a\cdot_{m}\overline{b+c}=\overline{a\cdot\overline{b+c}}\\
    &=\overline{a\cdot(b+c)}=\overline{a\cdot b+a\cdot c}\\
    &=\overline{a\cdot b}+\overline{a\cdot c}=a\cdot_{n}b+a\cdot_{n}c
  \end{aligned}$$
\end{pst}

\subsubsection{Polynomial Rings}
\begin{dft}
  Let $R$ be a nonzero ($1\neq 0$) commutative ring. A \textbf{polynomial} with coefficients in $R$ (in one variable) is a formal sum

  $$f(x)=\sum_{i=0}^{\infty}a_{i}x^{i}=a_{0}+a_{1}x+a_{2}x^{2}+\cdots+a_{n}x^{n}$$\s

  where $a_{i}\in R$ and $a_{i}\neq 0$ only for finitely many $i$.
\end{dft}\n

Note that a polynomial with coefficients in $R$ is essentially a finite sequence of elements in $R$.\n

\begin{dft}
  Let $f(x)$ be a polynomial defined in \rdft[\sctd{1}]. The \textbf{degree} of $f(x)$ is defined as

  $$\deg f(x):=\begin{cases}
    \max\brc{i\in\Z_{\geq 0}\srm a_{i}\neq 0}&\erm{if }f(x)\not\equiv 0\\
    -\infty&\erm{if }f(x)\equiv 0
  \end{cases}$$
\end{dft}\n

The notation $R[x]$ is the set of all polynomials with coefficients in $R$. With such notation, the following proposition can be introduced:\n

\begin{pst}
  Let $f(x)=\sum_{i=0}^{\infty}a_{i}x^{i}\in R[x]$ and $g(x)=\sum_{i=0}^{\infty}b_{i}x^{i}\in R[x]$ be polynomials defined in \rdft[\sctd{2}], then the following can be applied:

  \begin{alist}
    \item \textbf{Equality of Polynomials}\n

    $f(x)=g(x)$ if and only if $a_{i}=b_{i}$ for all $i$ as elements in $R[x]$.
    \item \textbf{Addition of Polynomials}\n

    $$f(x)+g(x)=\sum_{i=0}^{\infty}(a_{i}+b_{i})x^{i}$$

    \item \textbf{Multiplication of Polynomials}\n

    $$f(x)\cdot g(x)=\sum_{i=0}^{\infty}c_{i}x^{i}$$\s

    where $c_{i}=\sum_{k=0}^{i}a_{k}b_{i-k}$.
  \end{alist}
\end{pst}\n

Note that the addition operation is well-defined since $a_{i}+b_{i}\neq 0$ only for finitely many $i$. With the binary operations required, there is the following proposition:\n

\begin{pst}
  $(R[x],+,\cdot)$ is a nonzero commutative ring.\n

  \prf Note that the following are true:

  \begin{alist}
    \item $(R[x],+)$ is an abelian group with additive identity $0\in R[x]$, which is the zero polynomial.
    \item $1\in R[x]$ is the multiplicative identity.
    \item For associativity for multiplication, let $f(x)=\sum_{i}a_{i}x^{i},g(x)=\sum_{i}b^{i}x^{i},h(x)=\sum_{i}c^{i}x^{i}\in R[x]$, then

    $$\begin{aligned}[t]
      f\cdot(g\cdot h)&=\sum_{k=0}^{\infty}\brr{\sum_{i+j=k}a_{i}\brr{\sum_{m+n=j}b_{m}c_{n}}_{j}}x^{k}\\
      &=\sum_{k=0}^{\infty}\brr{\sum_{i+j=k}\sum_{m+n=j}a_{i}b_{m}c_{n}}x^{k}\\
      &=\sum_{k=0}^{\infty}\brr{\sum_{i+m+n=k}a_{i}b_{m}c_{n}}x^{k}
    \end{aligned}$$\s

    On the other hand,

    $$\begin{aligned}[t]
      (f\cdot g)\cdot h&=\brr{\sum_{l=0}^{\infty}\brr{\sum_{i+m=l}a_{i}b_{m}}x^{l}}\cdot\brr{\sum_{i}c_{i}x^{i}}\\
      &=\sum_{k=0}^{\infty}\brr{\sum_{l+n=k}\sum_{i+m=l}a_{i}b_{m}c_{n}}x_{k}\\
      &=\sum_{k=0}^{\infty}\brr{\sum_{i+m+n=k}a_{i}b_{m}c_{n}}x^{k}
    \end{aligned}$$

    \item The binary operation satisfies the distributive laws.
  \end{alist}
\end{pst}\n

\begin{dft}
  Given a polynomial $f(x)=\sum_{i=0}^{n}a_{i}x^{i}\in R[x]$, define a \textbf{function} as

  $$\phi_{f}:\alpha\in R\mapsto f(\alpha)=\sum_{i=0}^{n}a_{i}x^{i}\in R$$
\end{dft}\n

\begin{pst}
  Given polynomials $f,g\in R[x]$ with functions $\phi_{f},\phi_{g}:R\to R$, if $\phi_{f}=\phi_{g}$ as functions of $R\to R$, then $f=g$ as polynomials in $R[x]$.
\end{pst}

For example, $f(x)\in \R[x]$ defines a function $\phi_{f}:\R\to\R$ where $f(x)=x^{2}$. Moreover, if $g(x)\in \R[x]$ where $f(\alpha)=g(\alpha)$ for all $\alpha\in\R$, then $f=g$ in $R[x]$. This means that $f$ and $g$ have exactly the same coefficients.\n

However, this may not be true in general, especially when the cardinality of $R$ is finite. Consider $f=x^{2}+x+1\in\Z_{2}[x]$ and $g=x^{4}+x^{3}+x^{2}+x+1\in\Z_{2}[x]$, then $f\neq g$ in $\Z_{2}[x]$ but $f(\alpha)=g(\alpha)$ for all $\alpha\in\Z_{2}$.

\subsection{Integral Domains and Fields}
\subsubsection{Integral Domains}
\begin{dft}
  An \textbf{integral domain} is a nonzero commutative ring $R$ where the product of any two nonzero elements is always nonzero.
\end{dft}\n

\begin{dft}
  A nonzero element $r$ in a ring $R$ is called a \textbf{zero divisor} if there exists nonzero element $s\in R$ such that $rs=0$.
\end{dft}\n

With the definitions above, it can be shown that a nonzero commutative ring $R$ is an integral domain if and only if it has no zero divisors.\n

\begin{pst}
  A commutative ring $R$ is an integral domain if and only if the cancellation law holds for multiplication, which is $ca=cb$ and $c\neq 0$ implies $a=b$.\n

  \prf\arr Suppose $R$ is an integral domain. If $ca=cb$, then by distributive laws, $c(a-b)=c(a+(-b))=0$. Since $R$ is an integral domain, either $c=0$ and $a-b=0$. Given that $c\neq 0$, so $a=b$.\n

  \arl Suppose cancellation law holds. Suppose there are nonzero $a,b\in R$ such that $ab=0$, then by the previous result, $0=a0$ and this will give out $ab=a0$. Cancellation leads to $b=0$, which contradicts the assumption.
\end{pst}

\subsubsection{Units}
\begin{dft}
  Let $R$ be a ring, then an element $a\in R$ is said to be a \textbf{unit} if it has a multiplicative inverse. In other words, there exists $a^{-1}\in R$ such that $aa^{-1}=a^{-1}a=1$.
\end{dft}\n

\begin{exm}
  Below are various examples of units:

  \begin{alist}
    \item $1$ and $-1$ are the only units of $\Z$.
    \item Let $R$ be the ring of all real-valued functions on $\R$, then any function $f\in R$ satisfying $f(x)\neq 0$ for all $x$ is a unit.
    \item Let $R$ be the ring of all continuous real-valued functions on $\R$, then $f\in R$ is a unit if and only if it is either strictly positive or strictly negative.
  \end{alist}
\end{exm}\n

\begin{pst}
  The only units of $\Q[x]$ are nonzero constants.\n

  \prf Given any $f\in\Q[x]$ such that $\deg f>0$, for all nonzero $g\in\Q[x]$,

  $$\deg fg\geq\deg f>0=\deg 1$$\s

  and hence $fg\neq 1$. If $g=0$, then $fg=0\neq 1$, so $f$ has no multiplicative inverse.\n

  If $f$ is a nonzero constant, then $f^{-1}=\frac{1}{f}$ is a constant polynomial in $\Q[x]$ and

  $$f\brr{\frac{1}{f}}=\frac{1}{f}(f)=1$$\s

  Therefore $f$ is a unit.\n

  Finally, if $f=0$, then $fg=0\neq 1$ for all $g\in\Q[x]$, so the zero polynomial has no multiplicative inverse.
\end{pst}

\propdisp

\subsubsection{Fields}
\begin{dft}
  A \textbf{field} is a commutative ring with $1\neq 0$ where every nonzero element is a unit.
\end{dft}\n

In other words, a nonzero commutative ring $F$ is a field if and only if every nonzero element $r\in F$ has a multiplicative inverse $r^{-1}$, which is $rr^{-1}=r^{-1}r=1$.\n

\begin{exm}
  With the definition of fields, it can be shown that:

  \begin{alist}
    \item $\Q$, $\R$ and $\C$ are fields, but $\Z$ is not.
    \item Polynomial rings, including $\Q[x]$, $\R[x]$ and $\C[x]$, are not fields.
  \end{alist}
\end{exm}\n

Note that if every nonzero element of a commutative ring has a multiplicative inverse, then the ring is an integral domain since

$$ca=cb\Rightarrow c^{-1}ca=c^{-1}cb\Rightarrow a=b$$\s

and this leads to the following proposition:\n

\begin{pst}
  A field is an integral domain.
\end{pst}\n

\begin{pst}
  Let $k\in\Z_{m}\setminus\brc{0}$. If $\gcd(k,m)>1$, then $k$ is a zero divisor. If $\gcd(k,m)=1$, then $k$ is a unit.\n

  \prf Let $d=\gcd(k,m)$. If $d>1$, then $\frac{m}{d}$ is a nonzero element in $\Z_{m}$, and

  $$k\cdot_{m}\frac{m}{d}=\overline{\frac{k}{d}\cdot m}=0$$\s

  so $k$ is a zero divisor. On the other hand, if $d=1$, then there exists $a,b\in\Z$ such that $ak+bm=1$, but this means $\overline{a}k=1$ in $\Z_{m}$, so $k$ is a unit.
\end{pst}

% Fill missing

\subsection{Ring Homomorphisms}
\subsubsection{Definition of Ring Homomorphisms}
\begin{dft}
  Let $R$ and $R'$ be rings. A \textbf{ring homomorphism} from $R$ to $R'$ is a map $\phi:R\to R'$ such that the following are satisfied:

  \begin{alist}
    \item \textbf{Multiplicative Identity}

    $$\phi(1_{R})=1_{R'}$$

    \item \textbf{Addition}

    $$\phi(a+b)=\phi(a)+\phi(b)$$

    \item \textbf{Multiplication}

    $$\phi(ab)=\phi(a)\phi(b)$$
  \end{alist}
\end{dft}\n

Note that part (a) of the definition above is not imposed in some textbooks.

\subsubsection{Properties of Ring Homomorphisms}
\begin{pst}
  Let $R$ and $R'$ be rings. If $\phi:R\to R'$ is a ring homomorphism, then $\phi:(R,+)\to(R',+')$ is a homomorphism between abelian groups.
\end{pst}\n

In particular, $\phi(0_{R})=0_{R'}$ and $\phi(-a)=-\phi(a)$ for all $a\in R$. Note that the converse of the above proposition is not true. For example, $\psi:\Z\to\Z$ where $k\mapsto 2k$ is a group homomorphism, but it is not a ring homomorphism.\n

\begin{pst}
  Let $R$ and $R'$ be rings. If $\phi:R\to R'$ is a ring homomorphism and $u\in R^{x}$ is an element from the set of all units in $R$, then $\phi(u^{-1})=\phi(u)^{-1}$.\n

  \prf By the definition of ring homomorphism,

  $$\phi(u^{-1})\phi(u)=\phi(u^{-1}u)=\phi(1_{R})=1_{R'}$$\s

  then $\phi(u^{-1})=\phi(u)^{-1}$.
\end{pst}

\subsubsection{Examples of Ring Homomorphisms}
\begin{exm}
  Below are various examples of ring homomorphisms:

  \begin{alist}
    \item For any positive integer $n$, the remainder map $\phi:\Z\to\Z_{n}$ where $k\mapsto\overline{k}$ is a ring homomorphism, since the following conditions are satisfied:

    \begin{rlist}
      \item \textbf{Multiplicative Identity}

      $$\phi(1)=\overline{1}=1$$

      \item \textbf{Addition}

      $$\phi(k_{1}+k_{2})=\overline{k_{1}+k_{2}}=\overline{\overline{k_{1}}+\overline{k_{2}}}=\phi(k_{1})+\phi(k_{2})$$

      \item \textbf{Multiplication}

      $$\phi(k_{1}k_{2})=\overline{k_{1}k_{2}}=\overline{(\overline{k_{1}})(\overline{k_{2}})}=\phi(k_{1})\phi(k_{2})$$
    \end{rlist}

    Note that the zero map $\phi:R\to R'$ where $r\mapsto 0_{R'}$ is not a ring homomorphism because of part (a) of \rdft[\sctd{3}].

    \item For any ring $R$, the map $\phi:\Z\to R$ where

    $$n\mapsto n\cdot 1_{R}=\begin{cases}
      \underset{n\;\text{times}}{\underbrace{1_{R}+\cdots+1_{R}}}&\erm{if }n>0\\
      0_{R}&\erm{if }n=0\\
      \underset{n\;\text{times}}{\underbrace{(-1_{R})+\cdots+(-1_{R})}}&\erm{if }n<0
    \end{cases}$$\s

    is a ring homomorphism, since the following conditions are satisfied:

    \begin{rlist}
      \item \textbf{Multiplicative Identity}

      $$\phi(1)=1\cdot 1_{R}=1_{R}$$

      \item \textbf{Addition}

      $$\phi(n_{1}+n_{2})=(n_{1}+n_{2})1_{R}=n_{1}1_{R}+n_{2}1_{R}=\phi(n_{1})+\phi(n_{2})$$

      \item \textbf{Multiplication}

      $$\phi(n_{1}n_{2})=(n_{1}n_{2})1_{R}=(n_{1}1_{R})(n_{2}1_{R})=\phi(n_{1})\phi(n_{2})$$
    \end{rlist}

    Note that for any ring $R$, $\phi$ is the only ring homomorphism from $\Z$ to $R$. If $\theta:\Z\to R$ is a ring homomorphism, then $\phi(1)=1_{R}$ and this implies

    $$\begin{aligned}[t]
      \theta(n)&=\theta(\underset{n\;\text{times}}{\underbrace{1+\cdots+1}})\\
      &=\underset{n\;\text{times}}{\underbrace{\theta(1)+\cdots+\theta(1)}}\\
      &=n\theta(1)=n\cdot 1_{R}=\phi(n)
    \end{aligned}$$\s

    for all $n\in\Z$.
  \end{alist}
\end{exm}

\subsubsection{Characteristics of Integral Domains}
\begin{dft}
  Let $D$ be an integral domain. If there does not exist $n\in\Z_{>0}$ such that $\phi(n)=0$, then $D$ is said to have \textbf{characteristic} $0$, denoted as $\mathrm{char}(D)=0$.\n

  On the other hand, if there exists $n\in\Z_{>0}$ such that $\phi(n)=0$, then $D$ is said to have \textbf{positive characteristic}, and $\mathrm{char}(D)=\min\brc{n\in\Z_{>0}\srm n\cdot 1_{D}=0}$.
\end{dft}\n

\begin{pst}
  If $D$ is an integral domain of positive characteristic, then $\mathrm{char}(D)$ is a prime number.
\end{pst}\n

\begin{exm}
  Below are various examples of characteristics of integral domains:

  \begin{alist}
    \item $\Z,\Q,\R,\C$ are all of characteristic $0$.

    \item $\Z_{n}$ is not an integral domain if $n$ is composite, then $n=p$ where $p$ is a prime and $\mathrm{char}(\Z_{n})=p$.

    \item The \textbf{natrual inclusion}

    $$\phi:\Z\to\Q=\underset{\sim}{\underline{\Z\times(\Z\setminus\brc{0})}}$$\s

    where $n\mapsto\frac{n}{1}$ is a ring homomorphism.\n

    More generally, if $D$ is an integral domain, and

    $$F=\underset{\sim}{\underline{D\times(D\setminus\brc{0})}}$$\s

    is the field of fractions, then the map $\phi:D\to F$ where $a\mapsto[(a,1)]$ is an injective ring homomorphism, since the following conditions are satisfied:

    \begin{rlist}
      \item \textbf{Multiplicative Identity}

      $$\phi(1)=[(1,1)]$$

      \item \textbf{Addition}

      $$\phi(a_{1}+a_{2})=[(a_{1}+a_{2},1)]=[(a_{1},1)]+[(a_{2},1)]=\phi(a_{1})+\phi(a_{2})$$

      \item \textbf{Multiplication}

      $$\phi(a_{1}a_{2})=[(a_{1}a_{2},1)]=[(a_{1},1)][(a_{2},1)]=\phi(a_{1})\phi(a_{2})$$

      \item \textbf{Injectivity}\n

      To show that $\phi$ is injective,\m\m

      $$\ker\phi=\brc{a\in D\srm[(a,1)]=[(0,1)]}=\brc{a\in D\srm a\cdot 1=1\cdot 0}=\brc{0}\subset D$$
    \end{rlist}

    \item Let $R$ be a commutative ring. For any $a\in R$, the \textbf{evaluation map}

    $$\phi_{a}:R[x]\to R$$\s

    where $f(x)\mapsto f(a)$ is a surjective ring homomorphism.
  \end{alist}
\end{exm}

\subsection{Subrings and Ideals}
\subsubsection{Definition of Subrings}
\begin{dft}
  Let $R$ be a ring. A subset $S\subset R$ is called a \textbf{subring} if it is closed under the addition $+$ and multiplication $\cdot$ in $R$ and $S$ itself is a ring with unity $1_{R}$ under the induced operations. In other words, $S\subset R$ is a subring of $R$ if it satisfies the conditions in \rdft[\sctd{32}].
\end{dft}\n

Note that for any ring $R$, $\brc{0}\subset R$ is not a subring. Below is a practical criterion for checking whether a subset is a subring:\n

\begin{pst}
  Let $R$ be a ring, then a subset $S\subset R$ is a subring if and only if the following conditions are satisfied:

  \begin{alist}
    \item

    $$1_{R}\in S$$

    \item

    $$a-b\in S\;\forall a,b\in S$$

    \item

    $$a\cdot b\in S\;\forall a,b\in S$$
  \end{alist}
\end{pst}\n

\begin{pst}
  Let $\phi:R\to R'$ be a ring homomorphism, then the following are true:

  \begin{alist}
    \item If $S<R$ is a subring of $R$, then $\phi(S)<R'$.

    \item If $S'<R'$ is a subring of $R'$, then $\phi^{-1}(S')<R$.\n

    \prf Note that the following are true:

    \begin{rlist}
      \item

      $$\phi(1_{R})=1_{R'}\in S'\Rightarrow 1_{R}\in\phi^{-1}(S')$$

      \item Let $a,b\in\phi^{-1}(S')$, then $\phi(a),\phi(b)\in S'$.
    \end{rlist}

    Since $S'$ is a subring, then

    $$\begin{cases}
      \phi(a-b)=\phi(a)-\phi(b)\in S'\\
      \phi(ab)=\phi(a)\phi(b)\in S'
    \end{cases}\Rightarrow a-b,ab\in\phi^{-1}(S')$$\s

    hence $\phi^{-1}(S')$ is a subring of $R$.
  \end{alist}
\end{pst}\n

\begin{crl}
  For any ring homomorphism $\phi:R\to R'$, its image

  $$\mathrm{im}\:\phi=\phi(R)=\brc{\phi(a)\srm a\in R}$$\s

  is a subring of $R'$.
\end{crl}\n

Note that $\ker\phi$ is not a subring unless $R'$ is the zero ring, because $\phi(1_{R})=1_{R'}\neq 0$ unless $R'=\brc{0_{R'}}$.\n

\begin{exm}
  Consider an integral domain $D$ contained inside a field $F$, which is $D\subset F$. Let $\alpha\in F$, then this defines the evaluation map $\phi_{\alpha}:D[x]\to F$ where $f(x)\mapsto f(\alpha)$ is a ring homomorphism. By the corollary of \rpst[\sctd{1}],

  $$D[\alpha]:=\mathrm{im}\:\phi_{\alpha}=\brc{f(\alpha)\srm f\in D[x]}$$\s

  is a subring of $F$. In particular, $D[\alpha]$ is an integral domain.
\end{exm}

\subsubsection{Definition of Ideals}
\begin{dft}
  A subset $I$ of a commutative ring $R$ is said to be an \textbf{ideal} if the following properties are satisfied:

  \begin{alist}
    \item $0\in I$.
    \item If $a,b\in I$, then $a+b\in I$.
    \item For all $a\in I$, $ar\in I$ for all $r\in R$.
  \end{alist}
\end{dft}

\subsection{Quotient Rings}
\subsubsection{Residues}
\begin{dft}
  Let $R$ be a commutative ring and $I\subset R$ be an ideal. Consider the set $R/I=\brc{a+I\srm a\in R}$, then $a+I\in R\setminus I$ is said to be the \textbf{residue} of $a\in R$, denoted by $\overline{a}$.
\end{dft}

\subsubsection{Construction of Quotient Rings}
Recall that $\overline{a}$ is the equivalence class containing $a$ with respect to the relation $a\sim b\Leftrightarrow a-b\in I$. In particular, for any $a,b\in R$, $\overline{a}=\overline{b}\Leftrightarrow a-b\in I$. In this case, $a$ is said to be congruent to $b$ modulo $I$, denoted by $a\equiv b(\mathrm{mod} I)$.\n

Now define an operation $+$ on $R\setminus I$ such that

$$(a+I)+(b+I):=(a+b)+I$$\s

and the binary operation above is well-defined. More precisely, it has to be checked if

$$\begin{cases}
  a'+I=a+I\\
  b'+I=b+I
\end{cases}$$\s

then $(a'+b')+I=(a+b)+I$. Suppose

$$\begin{cases}
  a'+I=a+I\\
  b'+I=b+I
\end{cases}\Leftrightarrow\begin{cases}
  a'-a\in I\\
  b'-b\in I
\end{cases}$$\s

and since $(I,+)<(R,+)$,

$$\begin{aligned}[t]
  &(a'-a)+(b'-b)\in I\\
  \Rightarrow&(a'+b')-(a+b)\in I\\
  \Rightarrow&(a'+b')+I=(a+b)+I
\end{aligned}$$\s

so $+$ on $R\setminus I$ is well-defined. Furthermore, check if the binary operation can form an abelian group:\n

\begin{pst}
  $(R\setminus I,+)$ is an abelian group.\n

  \prf Note that associativity and commutativity of $+$ follow from those in $R$. The additive identity is $0+I=I$, and the additive inverse of $a+I$ is $(-a)+I$.
\end{pst}\n

Note that the above only involves the addition in $R$. In particular, given any subgroup $H$ in an abelian group $G$, the quotient abelian group $G\setminus H$ is constructed as above. If $\abs{G}<+\infty$, then

$$\abs{G\setminus H}=[G:H]=\frac{\abs{G}}{\abs{H}}$$\s

\begin{thm}
  Let $R$ be a commutative ring and $(I,+)<(R,+)$ is an additive subgroup, then the binary operation

  $$(a+I)\cdot(b+I):=(ab)+I$$\s

  is well-defined if and only if $I$ is an ideal.\n

  \prf\arl Suppose $I$ is an ideal. Also suppose that

  $$\begin{cases}
    a'+I=a+I\\
    b'+I=b+I
  \end{cases}$$\s

  then it has to be checked that $a'b'+I=ab+I$. Note that the above implies that there exists $i,j\in I$ such that

  $$\begin{cases}
    a'=a+i\\
    b'=b+j\\
  \end{cases}$$\s

  then

  $$a'b'=(a+i)(b+j)=ab+aj+bi+ij$$\s

  where $aj+bi+ij\in I$ since $I$ is an ideal. This means that $a'b'=ab$ and $a'b'+I=ab+I$.\n

  \arr Suppose the multiplication is well-defined. Let $a\in R$ and $i\in I$, then $i+I=0+I$ and the well-definedness of the multiplication implies that

  $$\begin{aligned}[t]
    (0+I)(a+I)&=(i+I)(a+I)\\
    I&=(ai)+I
  \end{aligned}$$\s

  so $ai\in I$ and $I$ is an ideal.
\end{thm}\n

\begin{crl}
  $(R\setminus I,+,\cdot)$ is a ring.\n

  \prf Associtativity of $\cdot$ and the distributive laws are inherited from $R$. The multiplicative identity is $1+I$.
\end{crl}

\subsubsection{Quotient Rings}
With the construction above, the definition of quotient rings is as below:\n

\begin{dft}
  Let $R$ be a commutative ring and $I\subset R$ be an ideal, then $R\setminus I$ is called the \textbf{quotient ring} of $R$ by the ideal $I$.
\end{dft}\n

\begin{pst}
  Consider the map $\pi:R\to R\setminus I$ where $a\mapsto a+I$, then $\pi$ is a surjective map homomorphism.\n

  \prf For any $a,b\in R$,

  $$\pi(a+b)=(a+b)+I=(a+I)+(b+I)=\pi(a)+\pi(b)$$\s

  and

  $$\pi(ab)=(ab)+I=(a+I)(b+I)=\pi(a)\pi(b)$$
\end{pst}

\subsubsection{First Isomorphism Theorem}
Below the \textbf{First isomorphism Theorem} is introduced:\n

\begin{thm}
  Let $\phi:R\to R'$ be a ring homomorphism. Then

  $$R\setminus\ker\phi\cong\mathrm{im}\phi$$\s

  More precisely, the map

  $$\overline{\phi}:R\setminus\ker\phi\to\mathrm{im}\phi\erm{where }a+\ker\phi\mapsto\phi(a)$$\s

  is a ring homomorphism such that $\phi=\overline{\phi}\circ\pi$.\n

  \prf \begin{alist}
    \item \textbf{Well-definedness of $\overline{\phi}$}\n

    Let $a,a'\in R$ such that $a+\ker\phi=a'+\ker\phi$, and this means that $\phi(a-a')=0$. With some evaluations, $\overline{\phi}(a+\ker\phi)=\phi(a)=\phi(a')=\overline{\phi}(a'+\ker\phi)$. Therefore $\overline{\phi}$ is well-defined.

    \item \textbf{Homomorphism Property of $\overline{\phi}$}\n

    Let $a+\ker\phi,b+\ker\phi\in R\setminus\ker\phi$, then

    $$\begin{aligned}[t]
      &\overline{\phi}((a+\ker\phi)+(b+\ker\phi))\\
      =&\overline{\phi}((a+b)+\ker\phi)\\
      =&\phi(a+b)=\phi(a)+\phi(b)\\
      =&\overline{\phi}(a+\ker\phi)+\overline{\phi}(b+\ker\phi)
    \end{aligned}$$\s

    and

    $$\begin{aligned}[t]
      &\overline{\phi}((a+\ker\phi)\cdot(b+\ker\phi))\\
      =&\overline{\phi}((ab)+\ker\phi)\\
      =&\phi(ab)=\phi(a)\phi(b)\\
      =&\overline{\phi}(a+\ker\phi)\overline{\phi}(b+\ker\phi)
    \end{aligned}$$

    \item \textbf{Isomorphism Property of $\overline{\phi}$}\n

    By definition, every element of $\mathrm{im}\phi$ is of the form $\phi(a)=\overline{\phi}(a+\ker\phi)$ for some $a\in R$. Then $\overline{\phi}$ is surjective. Also note that

    $$\begin{aligned}[t]
      \ker\overline{\phi}&=\brc{a+\ker\phi\srm\phi(a)=0}\\
      &=\brc{a+\ker\phi\srm a\in\ker\phi}\\
      &=\brc{\ker\phi}
    \end{aligned}$$\s

    therefore $\overline{\phi}$ is injective and $\overline{\phi}$ is an isomorphism for rings.
  \end{alist}

  Finally, for any $a\in R$,

  $$(\overline{\phi}\circ\pi)(a)=\overline{\phi}(a+\ker\phi)=\phi(a)$$\s

  therefore $\phi=\overline{\phi}\circ\pi$.
\end{thm}

In fact, $\pi$ is called the \textbf{canonical map} (or \textbf{projection map}).\n

\begin{exm}
  Below are various examples of quotient rings:

  \begin{alist}
    \item For any positive integer $n$, the remainder map

    $$\phi:\Z\to\Z_{n}$$\s

    is a surjective ring homomorphism with

    $$\ker\phi=\brc{k\in\Z\srm\overline{k}=0}=n\Z$$\s

    so by the First Isomorphism Theorem, $\Z\setminus n\Z\cong\Z_{n}$.

    \item Consider the map

    $$\phi:\Z\to\Z[i]\setminus(1+3i)\erm{where }n\mapsto\overline{n}=n+(1+3i)$$\s

    Note that $\Z[i]$ is also a PID, so every ideal in it is a principal ideal. $\phi$ is a surjective ring homomorphism and

    $$\ker\phi=10\Z$$\s

    \prf[zb] Let $m,n\in\Z$, then

    $$\phi(n+m)=(n+m)+(1+3i)=(n+(1+3i))+(m+(1+3i))=\phi(n)+\phi(m)$$\s

    and

    $$\phi(nm)=(nm)+(1+3i)=(n+(1+3i))(m+(1+3i))=\phi(n)\phi(m)$$\s

    To show that $\phi$ is surjective, observe that in $\Z[i]\setminus(1+3i)$, $\overline{1+3i}=\overline{0}$. Let $I=(1+3i)$, then

    $$\begin{aligned}[t]
      1+3i&\equiv 0(\mathrm{mod} I)\\
      1&\equiv-3i(\mathrm{mod} I)\\
      i&\equiv 3(\mathrm{mod} I)
    \end{aligned}$$\s

    For an arbitrary element $\overline{a+bi}\in\Z[i]\setminus(1+3i)$,

    $$\overline{a+bi}=\overline{a+3b}=\phi(a+3b)$$\s

    and hence $\phi$ is surjective. Now consider

    $$\ker\phi=\brc{n\in\Z\srm\overline{n}=\overline{0}}$$\s

    note that $\overline{n}=\overline{0}$ implies $n=(a-3b)+(b+3a)i$ for some $a+bi\in\Z[i]$. Solving the system of equations, $n=10a$, and the above argument shows that if $\overline{n}\in\ker\phi$, then $n\in 10\Z$ and $\ker\phi\subset 10\Z$. On the other hand, $10=(1+3i)(1-3i)$ leads to $=10\in\ker\phi$ and hence $10\Z\subset\ker\phi$.\n

    As a result, $\ker\phi=10\Z$. By the First Isomorphism Theorem, $\Z\setminus 10\Z\cong\Z[i]\setminus(1+3i)$.
  \end{alist}
\end{exm}

\subsection{Factorization of Polynomials}
\subsubsection{Principal Ideal Domains}
Recall an ideal $(a)=\brc{ar\srm r\in R}$ generated by one element $a\in R$ is called a principal ideal. Note that $R=(1)$ and $\brc{0}=(0)$ are both principal ideals.\n

\begin{dft}
  Let $R$ be an integral domain, then $R$ is a \textbf{principal ideal domain} (\textbf{PID}) if every ideal in $R$ is principal.
\end{dft}\n

Note that the definition of principal ideal domains brings out an interesting fact: any field is a principal ideal domain. However, the following proposition needs to be established first.\n

\begin{pst}
  Let $R$ be a commutative ring, then for all $d,f\in R[x]$ where the leading coefficient of $d$ is a unit in $R$, there exists $q,r\in R[x]$ such that

  $$f=qd+r,\;\deg(r)<\deg(d)$$\s

  \prf The proof is using induction to complete. For the base case, if $\deg(f)<\deg(d)$, take $r=f$, then $f=0d+r$ and $\deg(r)<\deg(d)$. For inductive step, let
  
  $$d=\sum_{i=0}^{n}a_{i}x^{i}\in R[x]$$\s
  
  be fixed where $a_{n}$ is a unit in $R$. For any given
  
  $$f=\sum_{i=0}^{m}b_{i}x^{i}\in R[x]$$\s

  where $m\geq n$, suppose the claim holds for all $f'$ with $\deg(f')<\deg(f)$, let

  $$f'=f-a_{n}^{-1}b_{m}x^{m-n}d$$\s

  By hypothesis there exists $q',r'\in R[x]$ with $\deg(r')<\deg(d)$ such that

  $$\begin{aligned}[t]
    f-a_{n}^{-1}b_{m}x^{m-n}d&=q'd+r'\\
    f&=(a_{n}^{-1}b_{m}x^{m-n}+q')d+r'
  \end{aligned}$$\s

  which is in the form $f=qd+r'$ where $q\in R[x]$ and $\deg(r')<\deg(d)$.
\end{pst}\n

\begin{thm}
  Let $F$ be a field, then $F[x]$ is a principal ideal domain.\n

  \prf Let $I$ be an ideal of $F[x]$ and $d$ be a nonzero polynomial in $I$ with the least leading degree. Such $d$ can be found since the leading degree of a polynomial is a nonnegative integer. $I$ is ideal implies $(d)\subset I$.\n

  By division theorem, $f=qd+r$ for some $q,r\in F[x]$ where $\deg(r)<\deg(d)$. Note that $r=f-qd$ lies in $I$. Since $d$ is a nonzero element of $I$ with the least degree, $r=0$, leaving $f=qd$. Finally, $f\in(d)$ and $I\subset(d)$, so $I=(d)$ can be concluded.
\end{thm}

\subsubsection{Factor Theorem for Polynomials}
\begin{dft}
  Let $F$ be a field and $f=\sum_{i=0}^{n}c_{i}x^{i}$ be a polynomial in $F[x]$. An element $a\in F$ is a \textbf{root} of $f$ if

  $$f(a)=\sum_{i=0}^{n}c_{i}a^{i}=0$$
\end{dft}\n

In fact the above definition is just a small recall from secondary school mathematics, but with the introduction of fields.\n

\begin{pst}
  For all polynomials $f\in F[x]$ and elements $a\in F$, there exists $q\in F[x]$ such that

  $$f=q(x-a)+f(a)$$
\end{pst}\n

With the proposition above, \textbf{Factor Theorem} can be introduced:\n

\begin{thm}
  Let $F$ be a field and $f$ be a polynomial in $F[x]$, then $a\in F$ is a root in $f$ if and only if $(x-a)$ divides $f$ in $F[x]$.
\end{thm}\n

\begin{thm}
  Let $F$ be a field and $f$ be a nonzero polynomial in $F[x]$ with degree $n$, then $f$ has at most $n$ roots in $F$. Moreover, if $a_{1},a_{2},\cdots,a_{n}\in F$ are distinct roots of $f$, then

  $$f=c(x-a_{1})(x-a_{2})\cdots(x-a_{n})$$\s

  where $c$ is a constant in $F$.
\end{thm}\n

\begin{crl}
  Let $F$ be a field, and $f,g$ be nonzero polynomials in $F[x]$ with maximum degree of $n$, or $n=\max\brc{\deg(f),\deg(g)}$, then if $f(a)=g(a)$ for $n+1$ distinct elements $a\in F$, then $f=g$.
\end{crl}

\subsubsection{Monic Polynomials}
\begin{dft}
  Let $f$ be a polynomial in $F[x]$, then $f$ is \textbf{monic} if its leading coefficient is $1$.
\end{dft}

\begin{pst}
  Let $F$ be a field, and $f,g$ be nonzero polynomials in $F[x]$, then there exists a unique monic polynomial $d\in F[x]$ with the following properties:

  \begin{alist}
    \item $(f,g)=(d)$.
    \item $d$ divides $f$ and $g$, or there exists $a,b\in F[x]$ such that $f=ad$ and $g=bd$.
    \item There exists $p,q\in F[x]$ such that $d=pf+qg$.
    \item If $h\in F[x]$ is a divisor of $f$ and $g$, then $h$ divides $d$.
  \end{alist}
\end{pst}\n

Note that the polynomial $d$ in the proposition above is called the \textbf{greatest common divisor} (\textbf{gcd}) of $f$ and $g$. If $d=1$, $f$ and $g$ are \textbf{relatively prime}.\n

\begin{dft}
  Let $p$ be a nonconstant polynomial in $F[x]$, then $p$ is \textbf{irreducible} if there do not exist $f,g\in F[x]$ such that $\deg(p)>\max\brc{\deg(f),\deg(g)}$ and $p=fg$.
\end{dft}

\subsubsection{Unique Factorization Domain}
\begin{dft}
  Let $D$ be an integral domain, then $D$ is a \textbf{unique factorization domain} (\textbf{UFD}) if any nonzero nonunit $r\in D$ can be factorized into a unique finite product of irreducible elements. 
\end{dft}\n

\begin{pst}
  If $F$ is a field, then $F$ is also a principal ideal domain and a unique factorization domain.
\end{pst}\n

\begin{pst}
  A polynomial $f\in F[x]$ is a unit if and only if it is a nonzero constant polynomial.\n

  \prf If $f,g\in F[x]$ are nonzero polynomials satisfying $fg=1$, then by comparing degrees on both sides gives $\deg(f)+\deg(g)=0$, which means $\deg(f)=\deg(g)=0$ and $f,g$ are constant polynomials.
\end{pst}\n

\begin{crl}
  Every nonconstant polynomial $f\in F[x]$ can be expressed as

  $$f=cp_{1}p_{2}\cdots p_{n}$$\s

  where $c$ is a nonzero constant, and each $p_{i}$ is a monic irreducible polynomial in $F[x]$. The factorization is unique up to reordering of the factors.
\end{crl}\n

With the proposition above, the greatest common divisor of two polynomials can be computed using the Euclidean Algorithm as in the case of $\Z$.\n

\begin{thm}
  Let $F$ be a field and $p$ be a polynomial in $F[x]$, then the following statements are equivalent:

  \begin{alist}
    \item $F[x]/(p)$ is a field.
    \item $F[x]/(p)$ is an integral domain.
    \item $p$ is irreducible in $F[x]$.
  \end{alist}
\end{thm}

\subsubsection{Rational Polynomials}
\begin{pst}
  Let $f=\sum_{i=0}^{n}a_{i}x^{i}$ be a polynomial in $\Q[x]$ with $a_{i}\in\Z$, then every rational root $r$ of $f$ in $\Q$ has the form $r=\frac{b}{c}$, where $\gcd\brc{b,c}=1$, $b\!\mid\!a_{0}$ and $c\!\mid\!a_{n}$.\n

  \prf Let $r=\frac{b}{c}$ be a root of $f$, then

  $$\sum_{i=0}^{n}a_{i}\brr{\frac{b}{c}}^{i}=0$$\s

  Multiply both sides with $c^{n}$ and rearrange gives

  $$a_{0}c^{n}=-b(a_{1}c^{n-1}+a_{2}c^{n-2}b+\cdots+a_{n}b^{n-1})$$\s

  and given that $b$ and $c$ are relatively prime, $b$ divides $a_{0}$. Similarly,

  $$a_{n}b^{n}=-c(a_{0}c^{n-1}+a_{1}c^{n-2}b+\cdots+a_{n-1}b^{n-1})$$\s

  shows that $c$ divides $a_{n}$.
\end{pst}\n

The proposition above is useful to determine whether a polynomial (especially with degree $3$ or greater) because such a polynomial is reducible only if it has a root in $\Q$.\n

\begin{exm}
  Check whether $f(x)=x^{3}+3x+2\in\Q[x]$ is reducible or not.\n

  \ans By \rpst[\sctd{1}], only $\pm 1$ and $\pm 2$ are possible roots. However, none of the possible roots satisfies the equation $f(x)=0$. Therefore, $f(x)$ is irreducible.
\end{exm}\n

\begin{dft}
  Let $f$ be a polynomial in $\Z[x]$, then $f$ is \textbf{primitive} if the greatest common divisor of its coefficients is $1$.
\end{dft}\n

By the definition above, a monic polynomial is primitive. Moreover, if $d$ is the greatest common divisor of coefficients of $f$, then $(1/d)f$ is primitive. Below is the \textbf{Gauss' Lemma}:\n

\begin{pst}
  If $f,g\in\Z[x]$ are both primitive, then $fg$ is also primitive.\n

  \prf Let $f=\sum_{k=0}^{m}a_{k}x^{k}$, $g=\sum_{k=0}^{n}b_{k}x^{k}$, then $fg=\sum_{k=0}^{m+n}c_{k}x^{k}$ where

  $$c_{k}=\sum_{i+j=k}a_{i}b_{j}$$\s

  Assume $fg$ is not primitive, then there exists a prime $p$ such that $p$ divides any $c_{k}$. Since $f$ is primitive, there exists a least $0<u<m$ such that $a_{u}$ is not divisible by $p$. Similarly, since $g$ is primitive, there exists a least $0<v<n$ such that $b_{v}$ is not divisible by $p$. Now

  $$c_{u+v}=\sum_{\substack{i+j=u+v\\(i,j)\neq(u,v)}}a_{i}b_{j}+a_{u}b_{v}$$\s

  Hence

  $$a_{u}b_{v}=c_{u+v}-\sum_{\substack{i+j=u+v\\i<u}}a_{i}b_{j}-\sum_{\substack{i+j=u+v\\j<v}}a_{i}b_{j}$$\s

  shows that every term on the right hand side are divisible by $p$. By Euclid's Lemma, $p$ divides either $a_{u}$ or $b_{v}$, which leads to a contradiction. Therefore, $fg$ is primitive.
\end{pst}\n

\begin{dft}
  Let $f$ be a nonzero polynomial in $\Q[x]$, then there exists a \textbf{content} of $f$, denoted by $c(f)$, and a primitive polynomial $f_{0}\in\Z[x]$ such that

  $$f=c(f)f_{0}$$
\end{dft}\n

Below are some propositions about contents of polynomials:\n

\begin{pst}
  If $f\in\Z[x]$, then $c(f)\in\Z$.\n

  \prf Let $d$ be the greatest common divisor of the coefficients of $f$, then naturally $(1/d)f$ is a primitive polynomial and

  $$f=d\brr{\frac{1}{d}f}$$\s

  is a factorization of $f$ into a product of a positive rational number and a primitive polynomial in $\Z[x]$. Therefore by the uniqueness of $c(f)$ and $f_{0}$, $c(f)=d\in\Z$.
\end{pst}\n

\begin{pst}
  Let $f,g,h$ be nonzero polynomials in $\Q[x]$ such that $f=gh$, then $c(f)=c(g)c(h)$ and $f_{0}=g_{0}h_{0}$.\n

  \prf The equation $f=gh$ implies

  $$c(f)f_{0}=c(g)c(h)g_{0}h_{0}$$\s

  where $c(f),c(g),c(h)$ are positive rational numbers, and $f_{0},g_{0},h_{0}$ are primitive polynomials. By Gauss' Lemma, $g_{0}h_{0}$ is also primitive. Finally, the uniqueness of content and primitive polynomial shows that $c(f)=c(g)c(h)$ and $f_{0}=g_{0}h_{0}$.
\end{pst}\n

\begin{thm}
  Let $f$ be a nonzero polynomial in $\Z[x]$. If $f=GH$ for some $G,H\in\Q[x]$, then $f=gh$ for some $g,h\in\Z[x]$, where $\deg(g)=\deg(G)$ and $\deg(h)=\deg(H)$.
\end{thm}\n

For the following theorem, let $p$ be a prime, and $\Z_{p}\cong\Z/\Z$. Note that $\Z_{p}$ is a field since $p$ is a prime. For $a\in\Z$, let $\overline{a}$ denote the residue of $a$ in $\Z_{p}$. 

\begin{thm}
  Let $f=\sum_{k=0}^{n}a_{k}x^{k}$ be a monic polynomial in $\Z[x]$. If $\overline{f}:=\sum_{k=0}^{n}\overline{a_{k}}x^{k}$ is irreducible in $\Z_{p}[x]$ for some prime $p$, then $f$ is irreducible in $\Q[x]$.
\end{thm}\n

Below is the \textbf{Einstein's Criterion}:\n

\begin{thm}
  Let $f=\sum_{k=0}^{n}a_{k}x^{k}$ be a polynomial in $\Z[x]$. If there exists a prime $p$ such that $p\!\mid\!a_{i}$ for $0\leq i<n$, but $p\!\nmid\!a_{n}$ and $p^{2}\!\nmid\!a_{0}$, then $f$ is irreducible in $\Q[x]$.
\end{thm}\n

\begin{exm}
  The polynomial $x^{5}+3x^{4}-6x^{3}+12x+3$ is irreducible in $\Q[x]$ by the Einstein's Criterion using $p=3$.
\end{exm}

\subsection{Field Extensions}
\subsubsection{Definition of Field Extensions}
\begin{dft}
  Let $E,F$ be fields, then $F$ is said to be a \textbf{subfield} if $F$ is a subring of $E$. In this case, $E$ is an extension of $F$, or $E/F$ is a \textbf{field extension}.
\end{dft}\n

Note that $E/F$ does not mean a quotient ring. Further let $\alpha$ be an element of $E$, then consider the evaluation map

$$\phi_{\alpha}:F[x]\to E,f\mapsto f(\alpha)$$\s

which is a homomorphism such that $\phi_{\alpha}\!\mid\!_{F}=\mathrm{id}_{F}$. The image of $\phi_{\alpha}$ is the subring

$$F[\alpha]:=\mathrm{im}\phi_{\alpha}=\brc{f(\alpha)\srm f\in F[x]}$$\s

in $E$. Since $E$ is a field, $F[\alpha]$ is an integral domain. Also, the subfield

$$F(\alpha)=\brc{\frac{f(\alpha)}{g(\alpha)}\srm f,g\in F[x],g(\alpha)\neq 0}$$\s

in $E$ is precisely the field of fractions of $F[\alpha]$. There are two scenarios:

\begin{nlist}
  \item $\ker\phi_{\alpha}=\brc{0}$, which means $\alpha$ is not a root of any nonzero polynomial $f\in F[x]$. In this case, $\alpha\in E$ is \textbf{transcendental} over $F$. Then $\phi_{\alpha}$ gives an isomorphism $F[x]\cong F[\alpha]$.
  \item $\ker\phi_{\alpha}\neq\brc{0}$, which means $\alpha$ is a root of some nonzero polynomial $f\in F[x]$. In this case, $\alpha\in E$ is \textbf{algebraic} over $F$. Since $F[x]$ is a principal ideal domain, $\ker\phi_{\alpha}=(p)$ for some $p\in F[x]$. By the First Isomorphism Theorem,
  
  $$\overline{\phi_{\alpha}}:F[x]/(p)\cong F[\alpha]$$\s

  Since $F[\alpha]$ is an integral domain, $p$ is irreducible and $\overline{\phi_{\alpha}}$ is a field. Therefore $\overline{\phi_{\alpha}}=F(\alpha)$ where $F(\alpha)$ is the smallest subfield of $E$ containing $F$ and $\alpha$. $F(\alpha)$ is said to be obtained from $F$ by \textbf{adjoining} $\alpha$.
\end{nlist}

\begin{thm}
  Let $E/F$ be a field extension and $\alpha$ be an element of $E$, then the following applies:

  \begin{alist}
    \item If $\alpha$ is algebraic over $F$, then $\alpha$ is a root of an irreducible polynomial $p\in F[x]$, such that $p\!\mid\!f$ for any $f\in F[x]$ with $f(\alpha)=0$.
    \item For $p$ be an irreducible polynomial $F[x]$ of which $\alpha$ is a root, then the map $\overline{\phi_{\alpha}}:F[x]/(p)\to F(\alpha)$ is defined by
    
    $$\phi\brr{\sum_{i=0}^{n}c_{i}x^{i}+(p)}=\sum_{i=0}^{n}c_{i}\alpha^{i}$$\s

    which is also a ring homomorphism mapping $x+(p)$ to $\alpha$ and $a+(p)$ to $a$ for any $a\in F$.
    \item Let $p$ be an irreducible polynomial in $F[x]$ of which $\alpha$ is a root. Then, each element in $F(\alpha)$ has a unique expression of the form
    
    $$c_{0}+c_{1}\alpha+\cdots+c_{n-1}\alpha^{n-1}$$\s

    where $c_{i}\in F$ and $n=\deg(p)$.
    \item If $\alpha,\beta\in E$ are both roots of an irreducible polynomial $p\in F[x]$, then there exists a ring homomorphism $\rho:F(\alpha)\to F(\beta)$ with $\rho(\alpha)=\beta$ and $\rho(s)=s$ for all $s\in F$.
  \end{alist}
\end{thm}

\subsubsection{Finite Fields}
Below is the \textbf{Kronecker's Theorem}:\n

\begin{thm}
  Let $F$ be a field, and $f$ be a nonconstant polynomial in $F[x]$, then there exists a field extension $E$ of $F$, such that $f\in F[x]\subset E[x]$ is a product of linear polynomials in $E[x]$. In other words, there exists a field extension $E$ of $F$ such that
  
  $$f=c(x-\alpha_{1})(x-\alpha_{2})\cdots(x-\alpha_{n})$$\s

  for some $c,\alpha_{i}\in E$.
\end{thm}\n

\begin{dft}
  Let $D$ be an integral domain, then the \textbf{characteristic} of $D$, denoted by $\mathrm{char}(D)$ is the smallest positive integer $n$ such that
  the sum of $1$ $n$ times is $0$. If the integer does not exist, $D$ has \textbf{characteristic zero}.
\end{dft}\n

\begin{pst}
  Let $F$ be a finite field, then the number of elements of $F$ is equal to $p^{n}$ for some prime $p$ and $n\in\N$.\n

  \prf Since $F$ is finite, it has finite characteristic, and since $F$ is a field, $\mathrm{char}(F)$ must be a prime $p$.
\end{pst}\n

With the proposition, below is the \textbf{Galois' Theorem}:\n

\begin{thm}
  Given any prime $p$ and $n\in\N$, there exists a finite field $F$ with $p^{n}$ elements.
\end{thm}\n

\begin{pst}
  Let $F$ be a field, and $f$ be a nonzero irreducible polynomial in $F[x]$, then $F[x]/(f)$ is a vector space of dimension $\deg(f)$ over $F$.
\end{pst}\n

\begin{crl}
  If $F$ is a finite field with $\abs{F}$ elements, and $f$ is a irreducible polynomial of degree $n$ in $F[x]$, then the field $F[x]/(f)$ has $\abs{F}^{n}$ elements.
\end{crl}

\pagebreak

\fancyhead[R]{References}
\section*{References}
\addcontentsline{toc}{section}{References}
The following are the references of the context of this document:


\begin{alist}
  \item Michael Artin, \textit{Algebra}, Pearson (2nd Edition), 2010.
  \item John B. Fraleigh, \textit{A First Course in Abstract Algebra}, Addison-Wesley (7th Edition), 2003.
\end{alist}

\end{document}
